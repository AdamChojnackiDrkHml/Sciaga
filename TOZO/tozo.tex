%%%% Uniwersalny Szablon Infy WPPT v4.2 (02.02.2018) %%%%

% uklad dokumentu
\documentclass[8pt]{extarticle}
\usepackage{xparse}
\usepackage[margin=0pt]{geometry}
\usepackage{enumerate} 
\frenchspacing
\linespread{0.5}
 \setlength{\parindent}{0pt}

% pakiety matematyczne
\usepackage{amssymb}   
\usepackage{amsthm}
\usepackage{amsmath}
\usepackage{amsfonts}
\usepackage{mathabx}

\usepackage{tikz}
\def\thm@space@setup{\thm@preskip=0pt
\thm@postskip=0pt}
% jezyk polski
\usepackage[polish]{babel}
\usepackage[utf8]{inputenc}
\usepackage{polski}
\usepackage[T1]{fontenc}

% hiperlacza
\usepackage{hyperref}
\hypersetup{
    colorlinks,
    citecolor=black,
    filecolor=black,
    linkcolor=black,
    urlcolor=black
}

\newtheoremstyle{mystyle}
  {1pt}%   Space above
  {1pt}% 
  {\slshape}%  Body font
  {}%          Indent amount (empty = no indent, \parindent = para indent)
  {\bfseries}% Thm head font
  {.}%         Punctuation after thm head
  {0.5em}%     Space after thm head: " " = normal interword space;
     %         \newline = linebreak
  {}%          Thm head spec (can be left empty, meaning `normal')

% wstawianie zdjec
\usepackage{graphicx} 

% deklaracja zadania
\theoremstyle{mystyle}
\newtheorem{zadanie}{Zadanie}[subsection]


% deklaracja metody
% \theoremstyle{remark}
\newtheorem*{metoda}{Metoda}
% \theoremstyle{plain}

% deklaracja rozwiazania
% \theoremstyle{remark}
\newtheorem*{rozwiazanie}{Rozwiązanie}
% \theoremstyle{plain}

\newtheorem{definicja}{Def}
\newtheorem{twierdzenie}{Tw}
\newtheorem{fakt}{F}

% zestaw - mpis
\newcommand{\EE}{\mathbb{E}}
\newcommand{\PP}{\mathbb{P}}
\newcommand{\Var}{\mathrm{Var}}
\newcommand{\DD}{\mathbb{D}}
\newcommand{\Cal}{\mathcal}
\newcommand{\bb}{\mathbb}
\newcommand{\red}[1]{\textcolor{red}{#1}}

% zestaw - programowanie
\usepackage{listings} 
\usepackage{minibox}
\usepackage{minted}
\usemintedstyle{borland}

% zestaw - akiso
%\usepackage{karnaugh-map} %%% nie ma w bazie overleaf
\usepackage{karnaughmap}
\usepackage{circuitikz}
\usepackage[inline]{enumitem}
\usepackage{tikz}
\usetikzlibrary{automata,positioning}

% zestaw - dyskretna
\newcommand{\stirf}[2]{\genfrac{[}{]}{0pt}{}{#1}{#2}}
\newcommand{\stirs}[2]{\genfrac{\{}{\}}{0pt}{}{#1}{#2}}

% zestaw - jftt
\newcommand{\oor}{\mathop{|}}
\DeclareMathOperator{\Lead}{Leading}
\DeclareMathOperator{\Trail}{Trailing}
\DeclareMathOperator{\fst}{First}
\DeclareMathOperator{\fol}{Follow}



\usepackage{multicol}
\usepackage{multirow}
\usepackage{float}
\setlength{\columnseprule}{1pt}
\def\columnseprulecolor{\color{black}}
\begin{document}

\begin{multicols*}{4}
    \begin{definicja}
    Deterministyczna maszyna Turinga $M$ to dwustronnie nieskończona taśma oraz czwórka $(Q, \Sigma, \delta, q_0)$, gdzie:
    \begin{itemize*}[label={}]
        \item $Q$ --- skończony zbiór stanów (bez \textbf{tak, nie h}),
        \item $\Sigma$ --- alfabet maszyny (poza $\sqcup$ domyślnym znakiem na maszynie),
        \item $\delta$ --- funkcja przejścia (wiadomo jak wygląda)
        \item $q_0 \in Q$ --- stan początkowy
    \end{itemize*}
\end{definicja}
\begin{definicja}
    Dla \textit{TM} $M$ i ciągu początkowego $x \in \Sigma^{*}$ przez $M(x)$
    oznaczamy wynik działania $M$ na $x$. W szczególności:
    \begin{itemize*}[label={}]
        \item $M(x) = tak$ --- maszyna kończy w stanie akceptującym,
        \item $M(x) = nie$ --- maszyna kończy w stanie odrzucającym,
        \item $M(x) = \nearrow$ --- maszyna nie zatrzymała się,
        \item $M(x) = y$ --- maszyna kończy pracę w \textbf{h} i $y \in (\Sigma \cup \{\sqcup\})^{*}$ jest zawartością taśmy bez blanków z lewej i prawej.
    \end{itemize*}
\end{definicja}
\begin{definicja}
    Konfiguracją maszyny $M$ będziemy nazywać trójkę $(q, \alpha, \beta)$,
    gdzie $\alpha, \beta \in {(\Sigma \cup \{\sqcup\})}^{*}$, 
    $\alpha$ symbole na taśme widziane przez maszynę bez zbędnych blanków po lewej,
    $\beta$ symbole na taśme widziane przez maszynę bez zbędnych blanków po prawej,
    $q$ aktualny stan maszyny.
    $\stackrel{M}{\rightarrow}$ definiujemy jako relację przejścia w jednym kroku między dwoma konfiguracjami maszyny $M$.
    $\stackrel{M^i}{\rightarrow}$ i $\stackrel{M^{*}}{\rightarrow}$ to odpowiednio przejście w $i$ krokach i przejście w dowolnej liczbie kroków.
\end{definicja}
\begin{definicja}
    $k$-taśmowa \textit{TM} ma funkcję przejścia w postaci:
    $\delta : \Cal{Q} \times {(\Sigma \cup \{\sqcup\})}^{k} \rightarrow$ $(\Cal{Q} \cup \{tak, nie, h\}) \times$ ${[(\Sigma \cup \{\sqcup\}) \times \{\leftarrow, \rightarrow, -\}]}^k$
\end{definicja}
\begin{twierdzenie}
    Dla dowolnej \textit{TM} $M$ z $k$ taśmami można skonstruować jednotaśmową \textit{TM} $M'$ taką, że:
    \begin{enumerate*}[label=\roman*)]
        \item dla każdego słowa wejściowego $x$ zachodzi $M(x) = M'(x)$ (maszyny są równoważne),
        \item jeśli $M$ na słowie wejściowym $x$ wykonała $f(|x|)$ kroków, to $M'$ wykona co najwyżej $O({(f(|x|))}^2)$.
    \end{enumerate*}
    Zwiększenie liczby taśmy, nie zwiększa możliwości obliczeniowych maszyn Turinga.
\end{twierdzenie}
\begin{twierdzenie}
    Dla dowolnej \textit{TM} $M$ pracującej w czasie $f(|x|)$ i dowolnego $\varepsilon > 0$
    można skonstruować równoważną \textit{TM} $M'$ pracującą w czasie $f'(|x|) = \varepsilon f(|x|) + |x| + 2$.
\end{twierdzenie}
\begin{definicja}
    Niedeterministyczna \textit{TM} ma funkcję przejścia w postaci:
    $\delta : \Cal{Q} \times {(\Sigma \cup \{\sqcup\})} \rightarrow$ $2^{(\Cal{Q} \cup \{tak, nie, h\}) \times {[(\Sigma \cup \{\sqcup\}) \times \{\leftarrow, \rightarrow, -\}]}}$
\end{definicja}
\begin{twierdzenie}
    Dla dowolnej jednostaśmowej \textit{NTM} $M$ można skonstruować dwutaśmową \textit{DTM} $M'$ taką, że:
    \begin{enumerate*}[label=\roman*)]
        \item Dla każdego słowa wejściowego $x$ zachodzi $M(x) = M'(x)$ (maszyny są równoważne).
        \item Jeśli $M$ na słowie wejściowym $x$ wykonała $f(|x|)$ kroków, to $M'$ wykona co najwyżej $O(c^{f(|x|)})$ kroków dla pewnej stałej $c$.
    \end{enumerate*}
    Niedeterminizm nie zwiększa możliwości obliczeniowych maszyn Turinga.
\end{twierdzenie}
\begin{definicja}
    Niech $L \subseteq \Sigma^*$. Niech \textit{TM} $M$, taka że dla każdego $x \in \Sigma^{*}$,
    jeżeli $x \in L$ to $M(x) = tak$,
    jeżeli $x \notin L$ to $M(x) = nie$.
    Wówczas mówimy, że $M$ rozstrzyga $L$ ($L$ nazywamy rozstrzygalnym/rekurencyjnym).
\end{definicja}
\begin{definicja}
    Niech $L \subseteq \Sigma^*$. Niech \textit{TM} $M$, taka że dla każdego $x \in \Sigma^{*}$,
    jeżeli $x \in L$ to $M(x) = tak$,
    Wówczas mówimy, że $M$ rozpoznaje $L$ ($L$ nazywamy rozpoznawalnym/rekurencyjnie przeliczalnym).
    Język rozpoznawany przez $M$ oznaczamy jako $L(M)$.
\end{definicja}
\begin{definicja}
    Niech $f : \Sigma^{*} \rightarrow {(\Sigma \cup \{\sqcup\})}^{*}$.
    Mówimy, że \textit{TM} $M$ oblicza $f$, jeżeli dla każdego $x \in \Sigma^{*}$ mamy $M(x) = f(x)$.
    $f$ nazywamy wtedy funkcją rekurencyjną.
\end{definicja}
\begin{lemat}
    Suma i przekrój języków rozpoznawalnych (rozstrzygalnych) jest językiem rozpoznawalnym (rozstrzygalnym).
\end{lemat}
\begin{lemat}
    Jeśli $L$ jest językiem rozstrzygalnym to $L'$ też.
\end{lemat}
\begin{lemat}
    $L$ jest rozstrzygalny $\Leftrightarrow$ $L$ i $L'$ są rozpoznawalne.
\end{lemat}
\begin{definicja}
    \textit{TM} $M$ nazywamy generatorem języka $L$, jeśli startując z pustym wejściem na specjalnej taśmie wyjściowej,
    wypisze wszystkie słowa z języka $L$ (każde słowo pojawi się na tej taśmie kiedyś).
    Język generowany przez $M$ oznaczamy jako $G(M)$.
\end{definicja}
\begin{lemat}
    Jeśli dla języka $L$ istnieje generator $G$, to $L$ jest rozpoznawalny.
\end{lemat}
\begin{lemat}
    Jeśli $M$ rozpoznaje $L$, to istnieje generator $G$ dla $L$.
\end{lemat}
\begin{lemat}
    Jeśli $L$ jest rozstrzygalny to istnieje generator wypisujący słowa z $L$
    w porządku leksykograficznym.
\end{lemat}
\begin{definicja}
    Właściwą funkcją złożoności nazywamy taką niemalejącą funkcję $f: \bb{N} \rightarrow \bb{N}$
    dla której istnieje \textit{DTM} która:
    \begin{enumerate*}[label=\roman*)]
        \item Startując z napisem $0^n$ na wejściu(read-only) kończy z napisem $0^{f(n)}$ na wyjściu(write-only).
        \item Pracuje w czasie $\leq O(n + f(n))$.
        \item Używa na taśmach roboczych $\leq O(f(n))$ komórek.
    \end{enumerate*}
\end{definicja}

    \begin{definicja}
    Niech $P=\{p_1, p_2, \dots, p_n\}$ będzie zbiorem $n$ graczy. Każdy z graczy $p_i, i \in [n]$,
    ma swój zwarty zbiór strategii (\textbf{czystych}) $S_i$.
    W każdej realizacji gry, każdy z graczy $p_i$ wybiera swoją strategię $s_i \in S_i$,
    tworząc w ten sposób wektor strategii $s = (s_1, s_2, \dots, s_n)$.
    Wtedy $S = \bigtimes_{n=1}^{n} S_i$ to zbiór możliwych \textbf{profili strategii}(wektorów strategii). 
    Każdy profil strategii $s$ jest jednznacznie wyznacza wektor wypłat $u(s) = (u_1(s), \dots, u_n(s))$ ($u: S \rightarrow V^n$, gdzie $V$ jest prz.l.),
    czyli zysków (lub strat) poszczególnych graczy. Każda z $u_i$ jest nazywana \textbf{funkcją wypłat} dla gracza $p_i$. 
    $(P, S, u)$ nazywamy \textbf{grą w postaci normalnej}. Jeśli $|A| < \aleph_0$, to mówimy, że gra jest skończona.
\end{definicja}

\begin{definicja}
    Niech $P = [n]$. Dla danego profilu strategii $s \in S$ określamy $s_{-i}$,
    jako profil strategii $(s_1, \dots, s_{i-1}, s_{i+1}, \dots, s_n)$ z pominięciem strategii czystej $i$-tego gracza.
    Będziemy wtedy zapisywać skrótowo $s = \langle s_i, s_{-i} \rangle$.
    Mówimy, że statega czysta $s_{i}^{*}$:
    \begin{enumerate*}[label=\roman*)]
        \item \textbf{słabo dominuje} $s_{i}'$ jeśli $(\forall s \in S)(u_i(\langle s_{i}^{*}, s_{-i} \rangle) \geq u_i(\langle s_{i}^{'}, s_{-i} \rangle))$
        \item \textbf{silnie dominuje} $s_{i}'$ jeśli $(\forall s \in S)(u_i(\langle s_{i}^{*}, s_{-i} \rangle) > u_i(\langle s_{i}^{'}, s_{-i} \rangle))$
        \item \textbf{najlepszą odpowiedzią na} $s_{-i}$ $(\forall s_i \in S_i)(u_i(\langle s_{i}^{*}, s_{-i} \rangle) \geq u_i(\langle s_{i}, s_{-i} \rangle))$
    \end{enumerate*}
    Mówimy, że profil strategii $s^{*} = (s_{1}^{*}, \dots, s_{n}^{*}) \in S$:
    \begin{enumerate*}[label=\roman*)]
        \item jest \textbf{strategią słabo dominującą}: $(\forall i \in [n])(\forall s \in S)(u_i(\langle s_{i}^{*}, s_{-i} \rangle) \geq u_i(s))$
        \item jest \textbf{strategią silnie dominującą}: $(\forall i \in [n])(\forall s \in S, s_i \neq s_{i}^{*})(u_i(\langle s_{i}^{*}, s_{-i} \rangle)$ $> u_i(s))$
        \item jest \textbf{równowagą Nasha w strategiach czystych}: $(\forall i \in [n]) (s_{i}^{*}$ jest najlepszą odpowiedzią na $s_{-i}^{*})$.
        \item jest \textbf{optimum Pareta}: $(\forall s \in S\setminus{s^{*}})(\{[(\exists j \in [n])u_{j}(s) < u_{j}(s^{*})]$ $\lor [u(s) = u(s^{*})]\})$.
    \end{enumerate*}
    Zbiór wszystkich optimum Pareta, jest nazywany \textbf{frontem Pareta}.
\end{definicja}
\begin{fakt}
    Dla każdego gracza, słaba dominacja w sensie strategii czystych jest relacją częsciowego porządku.
\end{fakt}
\begin{fakt}
    $s \in S$ jest rowiązaniem (słabym) $\Leftrightarrow$ $(\forall p_i \in P)$ ($s_i$ słabo dominuje wszystkie inne strategie tego gracza).
\end{fakt}
\begin{fakt}
    Jeśli $s \in S$ jest rozwiązaniem słabym, to dla każdego gracza $p_i$, $s_i$ jest najlepszą odpowiedzią,
    na każdy możliwy profil strategii innych graczy $s_{-i}'$. W szczególności jest równowagą Nasha dla strategii czystych.
\end{fakt}
\begin{fakt}
    Jeśli istnieją strategie czyste ściśle zdominowane, usuwaj je, dopóki się da. 
    W wyniku takiego algorytmu otrzymamy uproszczoną grę.
    Nawet gdy dojdziemy do gry z jedną strategią, to nie musi być to strategia dominująca nawet w słabym sensie.
\end{fakt}
\begin{twierdzenie}
    Jeśli istnieje rozwiązanie ściśle dominujące, to jest ono jedno i algorytm eliminacji prowadzi do tego rozwiązania
\end{twierdzenie}
\begin{definicja}
    \textbf{Gra koordynacyjna} to gra, w której gracz ma identyczny zbiór strategii czystych oraz wypłata gracza
    $p \in P$ jest niemalejącą funkcją ze względu na rosnącą liczbę graczy, która wybierze tę samą czystą strategię. 
    \textbf{Gra anty-koordynacyjna} analogicznie ma nierosnącą funkcję.
    \textbf{Gra dysokoordynacyjna} dla dwóch osób, jeden z graczy dąży do koordynacji, drugi  do antykoordynacji strategii czystych.
\end{definicja}
    \begin{definicja}
    Jeśli gra $G$ spełnia warunek: $(\forall s \in S)(\sum\limits_{i \in [n]}u_i = 0)$,
    to taką grę nazywamy \textbf{gra o sumie zerowej}. 
    Jeśli $K$ jest ciałem skalarów nad $V$, to grę $G$, która spełnia warunek 
    $(\exists c \in K)(\forall s \in S)((\sum\limits_{i \in [n]}u_i = c))$
    nazywamy \textbf{grą o sumie stałej}.
\end{definicja}
\begin{fakt}
    Każda gra o sumie zerowej jest grą o sumie stałej.
\end{fakt}
\begin{twierdzenie}
    Każda gra o sumie niezerowej na $n$ graczy, jest równoważna z grą o sumie zerowej dla $n+1$ graczy.
\end{twierdzenie}
\begin{fakt}
    W grzez o sumie stałej, każda strategia jest Pareto-optymalna.
\end{fakt}
\begin{definicja}
    Rozważmy grę dwuosobową o sumie stałej $c$, w której $S_1, S_2$ są zwartymi podzbiorami $\bb{R}^k$
    (dla pewnego $k \in \bb{N}$).
    \textbf{Wartością dolną} gry $\underline{v}$ nazywamy $\sup_{s_1 \in S_1} \inf_{s_2 \in S_n} u_1(s_1, s_2)$.
    \textbf{Wartością górną} gry $\overline{v}$ nazywamy $\inf_{s_1 \in S_1} \sup_{s_2 \in S_n} u_1(s_1, s_2)$.
    Jeśli $\overline{v} = \underline{v}$, to mówimy, że gra ma \textbf{wartość} $v$.
    Profil strategii, który realizuje wartość gry, nazywamy \textbf{punktem siodłowym}.
    Profil strategii $(s_1, s_2)$ taki, że $s_1$ realizuje wartość dolną, a $s_2$ górną, nazywamy \textbf{strategią minimaksową/maksiminową}
\end{definicja}
\begin{twierdzenie}
    Dla gry dwuosobowej o sumie stałej:
    \begin{enumerate*}[label=\roman*)]
        \item $\underline{v} \leq \overline{v}$,
        \item Gra ma równowagę Nasha $\Rightarrow$ $\underline{v} = \overline{v}$ i każdy punkt siodłowy jest równowagą Nasha.
    \end{enumerate*}
\end{twierdzenie}
\begin{definicja}
    Profile strategii, które mają wartość z przedziału $[\underline{v}, \overline{v}]$, nazywamy \textbf{strategiami bezpieczeństwa}.
\end{definicja}
\begin{definicja}
    Niech $(P = [n], S, u)$, będzie grą skończoną w postaci normalnej.
    \textbf{Strategią mieszaną gracza} $i$ nazywamy rozkład prawdopodobieństwa $\pi_i : S_i \rightarrow [0, 1]$.
    \textbf{Strategią mieszaną gry} nazywamy rozkład produktowy $\pi := \bigtimes_{i=0}^{n}\pi_{i} : S \rightarrow [0, 1]$.
    Wypłatą gracza $j$ jest wtedy wartość oczekiwana: $u_j(\pi) = \sum\limits_{(s_j \in S_j)}\sum\limits_{s_{-j} \in S_{-j}} u_j(\langle s_j, s_{-j} \rangle)\pi(\langle s_j, s_{-j} \rangle)$.
    Zbiór wszystkich rozkładów gracza $i$ będziemy oznaczać przez $\Delta_i$, a ich produkt przez $\Delta$.
\end{definicja}
\begin{definicja}
    \textbf{Mieszaną równowagą Nasha} nazywamy rozkład łączny $\pi^{*}$ taki, że:
    $(\forall i \in [n])(\forall \pi_i \in \Delta_i)(u_i(\pi^{*}) \geq u_i(\langle \pi_i, \pi_{-i}^{*} \rangle))$.
\end{definicja}
\begin{twierdzenie}
    Każda czysta równowaga Nasha $s^{*} \in S$ jest mieszaną równowagą Nasha.
\end{twierdzenie}
\begin{definicja}
    \textbf{Nośnikiem} rozkładu łącznego strategii mieszanej $\pi$ nazywamy zbiór
    $S(\pi) = \{s \in S : \pi(s) > 0\}$.
    \textbf{Nośnikiem} rozkładu strategii mieszanej gracza $\pi_i$ nazywamy zbiór
    $S(\pi_i) = \{s_i \in S_i :\pi_i(s_i) > 0\}$.
\end{definicja}
\begin{twierdzenie}
    Niech $\pi^{*} \in \Delta$ będzie \textbf{MNE}.
    Wtedy $(\forall s_i \in S(\pi_{i}^{*}))(u_i(\langle s_i, \pi_{-i}^{*} \rangle) = u_i(\pi))$
\end{twierdzenie}
    \begin{definicja}
    Załóżmy, że $A$ jest problemem optymalizacyjnym. Niech $F_A(x)$ - zbiór dopuszczalnych rozwiązań, $c(s)$ dla $s \in F(x)$ - dodatni całkowitoliczbowy koszt rozwiązana $s$, $OPT_A(x) = min_{s \in F(x)} c(s)$ - koszt rozwiązania optymalnego. Algorytm $M$ jest algorytmem $\epsilon$-aproksymacyjnym (dla $1 > \epsilon \geq 0$) problemu $A$, jeśli 
    \begin{enumerate*}[label=\roman*)]
        \item $M$ jest algorytmem wielomianowym,
        \item $M(x) \in F_A(x)$,
        \item $\frac{|c(M(x)) - OPT_A(x)|}{max\{OPT_A(x), c(M(x))\}} \leq \epsilon$
    \end{enumerate*}
\end{definicja}

\begin{definicja}
    Próg aproksymacji problemu $A$ to największe dolne ograniczenie wszystkich wartości $\epsilon > 0$ dla których istnieje algorytm $\epsilon$-aproksymacyjny.
\end{definicja}

\begin{definicja}
    $NODE$ $COVER$ --- dla grafu $G = (V,E)$ chcemy znaleźć najmniejszy ze względu na moc $C \subseteq V$ taki, że $\forall_{(u,v) \in E} u \in C \lor v \in C$.
\end{definicja}

\begin{twierdzenie}
    Próg aproksymacyjny problemu $NODE$ $COVER$ jest nie większy niż $\frac{1}{2}$. 
\end{twierdzenie}

\begin{definicja}
    $MAX$ $CUT$ --- dla grafu $G = (V,E)$ taki podział $V$ na dwa zbiory $S$ i $V \setminus S$ aby liczba krawędzi między nimi była jak największa.
\end{definicja}

\begin{lemat}
    Rozwiązanie wygenerowane przez lokalne polepszenia dla problemu $MAX$ $CUT$ ma co najmniej połowę optymalnej liczby krawędzi.
\end{lemat}

\begin{twierdzenie}
    Próg aproksymacyjny problemu $MAX$ $CUT$ jest nie większy niż $\frac{1}{2}$. 
\end{twierdzenie}

\begin{twierdzenie}
    Jeżeli $P \neq NP$, to próg aproksymacyjny problemu komiwojażera wynosi $1$ (oznacza to, że problem nie jest aproksymowalny).
\end{twierdzenie}

\begin{definicja}
    Wielomianowy schemat aproksymacji (PTAS) dla problemu optymalizacyjnego $A$ to algorytm, który dla każdego $\epsilon > 0$ i przykładu $x \in A$ zwraca rozwiązanie o błędzie względnym równym co najwyżej $\epsilon$, w czasie wielomianowym zależnym od $\epsilon$ i długości $x$. Schemat nazywamy w pełni wielomianowym (FPTAS), gdy tylko współczynniki wielomianu zależą od $\frac{1}{\epsilon}$.
\end{definicja}

\begin{twierdzenie}
    Próg aproksymacji dla problemu plecakowego wynosi $0$. Ma on w pełni wielomianowy schemat aproksymacji (FPTAS).
\end{twierdzenie}
    \begin{definicja}
    \textbf{Przekształceniem afinicznym} wypłat nazywamy $u^{\prime} = \varphi(u)$ zadane wzorem $(\forall i \in [n])(\forall s \in S)u_i^{\prime}(s) = \alpha_i u_i(s) + \beta_i(s_{-i}),$ gdzie $\alpha_i > 0, \beta_i : S_{-i} \rightarrow \mathbb{R}.$ 
\end{definicja}

\begin{twierdzenie}
    Jesli $\varphi$ jest przekształceniem afinicznym, to gry z wypłatami u i $u^{\prime} = \varphi(u)$ mają te same równowagi Nasha (lecz ze zmienionymi wypłatami).
\end{twierdzenie}

\begin{definicja}
    Grę nazywamy \textbf{generyczną}, jeśli $(\exists \epsilon > 0)(\forall s \in S)(\forall i \in [n])$, zmiana $u_i(s)$ o $\epsilon$, nie zmienia równowag Nasha.
\end{definicja}

\begin{definicja}
    \textbf{Grą jednoosobową} nazywamy grę, w której $P$ składa się z gracza $p$ oraz „natury”. Jest to dwuosobowa gra w postaci normalnej o sumie zerowej, gdzie strategie natury wybierane są losowo.
\end{definicja}

\begin{definicja}
    \textbf{Równowagą skorelowaną} (strategią skorelowaną) nazywamy strategię mieszaną $\pi^{\star}$
    (niekoniecznie o niezależnych współrzędnych (tj. rozkładach brzegowych)) zaproponowaną przez zaufaną wyrocznię, której dowolna realizacja $\sigma$ spełnia warunek: $(\forall i \in [n])(\forall \pi_i \in \Delta_i) \sum_{s_{-i} \in S_{-i}(\pi^{\star})} (u_i(s) - u_i(\langle \pi_i, s_{-i} \rangle)) \mathbb{P}[\sigma = s | \sigma_i = s_i] > 0.$
\end{definicja}

\begin{twierdzenie}
    Każda równowaga Nasha jest równowagą skorelowaną. 
\end{twierdzenie}


\begin{twierdzenie}
    W grach dwuosobowych, równowagi Nasha są wierzchołkami wieloscianów wyznaczonych przez nierówności dla równowag skorelowanych.
\end{twierdzenie}
    \begin{definicja}
    \textbf{Algorytm szukania równowag Nasha}
    \begin{enumerate*}[label=\roman*)] 
        \item Wykreśl strategie (ściśle) zdominowane. 
        \item Heurystycznie należy wyznaczyć nośnik równowag Nasha, rozwiązując układy równań, opartych o tw. o równości wypłat.
        \item Sprawdź, czy dają one równowagę Nasha.
    \end{enumerate*}
\end{definicja} 

\begin{twierdzenie}
    \textbf{Lemke - Howson}. Niech $z = (x $++$ y)^T \in \mathcal{M}_{(m+n) \times 1}, \mathbbm{1}_{m+n} = (1,\dots,1)^T \in \mathcal{M}_{(m+n) \times 1}$ oraz niech $C = \begin{bmatrix} 0 & A \\ B^T & 0 \end{bmatrix}$, gdzie wartości macierzy $A$ oraz $B$ są dodatnie. Wtedy dowolne niezerowe rozwiązanie $z^{\star}$ problemu: $z \geq 0; \mathbbm{1}_{m+n} - Cz \geq 0; x_T (\mathbbm{1}_{m+n} - Cz ) = 0$ wyznacza $\pi = \left(\frac{x}{\sum_{i=1}^m x_i} ,\frac{y}{\sum_{j=1}^n y_j}\right)$, które jest równowagą Nasha w grze dwumacierzowej z wypłatami zadanymi przez macierze $A$ i $B$ i vice versa, każda równowaga Nasha jest rozwiązaniem tego problemu.
\end{twierdzenie}

\begin{definicja}
    \textbf{Algorytm Lemkego-Howsona}
    \begin{enumerate*}[label=\roman*)] 
        \item Niech $z$ będzie rozwiązaniem zerowym, kładziemy $D|r = -C|\mathbbm{1}_{m+n}^T$ (Niech $D = [d_{ij}]$) oraz ustawiamy liczbę kroków $K = 1$.
        \item Wybieramy dowolną kolumnę $j_K$ macierzy $D$ i szukamy $d_{i_K} = min_i d_{ij_K}$.
        \item Podmieniamy elementy macierzy D tak, żeby:
        \begin{enumerate*} 
            \item $d_{i_K j_K} \leftarrow \frac{1}{d_{i_K j_K}}$,
            \item $d_{i_K j} \leftarrow \frac{d_{i_K j}}{d_{i_K j_K}}$, gdy $j \neq j_K$,
            \item $r_{i_K} \leftarrow \frac{r_{i_K}}{d_{i_K j_K}}$,
            \item $d_{ij_K} \leftarrow - \frac{d_{ij_K}}{d_{i_K j_K}}$, gdy $i \neq i_K$,
            \item $d_{ij} \leftarrow d_{ij} - \frac{d_{i_K j} d_{ij_K}}{d_{i_K j_K}}$, gdy $i \neq i_K \land j \neq j_K$,
            \item $r_{i} \leftarrow r_i - \frac{r_{i_K} d_{ij_K}}{d_{i_K j_K}}$, gdy $i \neq i_K$.
        \end{enumerate*}
        \item Jeśli $j_1 = i_K$ to kończymy algorytm.
        \item W przeciwnym razie $K \leftarrow K + 1$, wstawiamy $j_K = i_{K-1}$, szukamy $d_{i_K} = min_i d_{ij_K}$ i wracamy do kroku 3.
    \end{enumerate*}
    Gdy w K-tym kroku otrzymaliśmy $i_K$ i $j_K$, to oznacza, że w rozwiązaniu $z_{j_K} = r_{i_K}$, gdzie wartość ta jest brana z końcowego wektora $r$.
\end{definicja} 

    \begin{definicja}
    Problem \textbf{QSAT} to problem prawdziwości formuł logicznych z kwantyfikatorami $Q_1,\ldots,Q_n$, bez zmiennych wolnych (zdań logicznych). $\exists_{x_1}\forall_{x_2}\exists_{x_3}\ldots Q_{nx_n}\phi(x_1,\ldots,x_n)$
\end{definicja}

\begin{twierdzenie}
    \textbf{QSAT} jest \textbf{PSPACE}-zupełny.
\end{twierdzenie}

\begin{definicja}
    \textbf{Alternująca maszyna Turinga} to niedeterministyczna maszyna $M=(Q_{\exists}\cup Q_{\forall},\Sigma,\delta,q_0)$. 
    Aby zaakceptować obliczenie dla stanów z $Q_{\exists}$ wymagamy istnienia od nich ścieżki akceptującej.
    Dla stanów z $Q_{\forall}$ - wszystkie ścieżki akceptujące.
\end{definicja}

\begin{definicja}
    \textbf{ATIME(f)} to zbiór wszystkich dokładnych, alternujących maszyn Turinga pracujących w czasie co najwyżej $f$. \\
    \textbf{ASPACE(f)} to zbiór wszystkich dokładnych, alternujących maszyn Turinga pracujących z pamięcią co najwyżej $f$.
\end{definicja}

\begin{definicja}
    $AL=ASPACE(logn),AP=\Cup_{j\in \mathbb{N}}NTIME(n^j)$
\end{definicja}

\begin{twierdzenie}
    $AP=PSPACE$
\end{twierdzenie}

\begin{twierdzenie}
    $QSAT$ jest $AP$-zupełny.
\end{twierdzenie}

\begin{definicja}
    \textbf{Monotone Circuit Value} to problem obliczenia wartościowania sieci boolowskiej złożonej tylko z bramek \textit{AND} i \textit{OR}.
\end{definicja}

\begin{twierdzenie}
    \textbf{Monotone Circuit Value} jest $P$-zupełny. \\
    \textbf{Monotone Circuit Value} jest $AL$-zupełny. \\
    $AL=P$ \\
    $ASPACE(f(n))=TIME(k^{f(n)})$
\end{twierdzenie}

\begin{definicja}
    Modyfikacja ATM - aby odczytać zawartość $i$-tej komórki, wystarczy napisać jej adres na specjalnej taśmie roboczej i wejść w odpowiedni stan.
    Dzięki temu możemy rozpatrywać obliczenia w czasie podliniowym.
\end{definicja}



    \begin{definicja}
    Rozważamy boolowskie funkcje logiczne $\{0,1\}^n\rightarrow\{0,1\}$, wyrażone w postaci sieci logicznych, jako kodowanie problemów decyzyjnych.
    \textbf{Rozmiar} sieci logicznej to liczba bramek logicznych. 
    \textbf{Głębokość} sieci to długość najdłuższej ścieżki od któregoś wejścia do wyjścia.
\end{definicja}

\begin{definicja}
    \textbf{Rodzina} sieci to nieskończony ciąg $C=\{C_1,C_2\ldots\}$ sieci logicznych, gdzie $C_n$ ma $n$ zmiennych wejściowych.
\end{definicja}

\begin{definicja}
    Język $L\subseteq\{0,1\}^*$ ma \textbf{sieci wielomianowe}, jeżeli istnieje rodzina sieci $C=\{C_1,C_2,\ldots\}$ taka, że
    1) rozmiar $C_n$ jest równy co najwyżej wartości wielomianu $p(n)$;
    2) $\forall_{x\in\{0,1\}^*}{x\in L \iff C_{|x|}(x)=1}$.
\end{definicja}

\begin{lemat}
    Każdy język z klasy $P$ ma sieć wielomianową.
\end{lemat}

\begin{lemat}
    Istnieją języki nierozstrzygalne posiadające sieci wielomianowe.
\end{lemat}

\begin{definicja}
    Rodzinę sieci $C=\{C_1,C_2\ldots\}$ nazywamy \textbf{jednostajną}, jeżeli istnieje maszyna Turinga $N\in L$ taka, że 
    dla wejścia $1^n$ generuje sieć $C_n$ (sieć musi mieć wielomianowy rozmiar w stosunku do $n$).
\end{definicja}

\begin{twierdzenie}
    Język $L$ ma jednostajną rodzinę sieci $\iff L\in P$.
\end{twierdzenie}

\begin{definicja}
    Niech $C$ jednostajna rodzina sieci i $f(n), g(n):\mathbb{N}\rightarrow\mathbb{N}$.
    Mówimy, że \textbf{równoległy czas} rodziny $C$ wynosi $f(n)$, jeżeli dla każdego $n$ głębokość sieci $C_n$ nie przekracza $f(n)$.
    Mówimy, że \textbf{całkowita praca} rodziny sieci $C$ wynosi $g(n)$, jeżeli dla każdego $n$ rozmiar sieci $C_n$ nie przekracza $g(n)$.
\end{definicja}

\begin{definicja}
    $PT/WK(f(n), g(n))$ to klasa wszystkich języków $L\in\{0,1\}^*$ dla których istnieje jednostajna rodzina sieci $C$ 
    rozstrzygająca $L$ w czasie $O(f(n))$ i wykonująca pracę $O(g(n))$.
\end{definicja}

\begin{lemat}
    Jeżeli $L\in PT/WK(f(n),g(n))$, to istnieje maszyna Turinga $M\in L$ generująca program na maszynę $PRAM$
    obliczającą $L$ w czasie $O(f(n))$ przy użyciu $O(g(n)/f(n))$ procesorów.
\end{lemat}

\begin{definicja}
    Klasa \textbf{NC} - równoległe obliczenia w czasie polilogarytmicznym (z pracą wielomianową).
    $NC_i = PT/WK(log^in, n^k)$
    $NC = \bigcup_{i\in\mathbb{N}}{NC_i}$
\end{definicja}


    \begin{definicja}
    Niech $Q$ będzie wielomianowo zrównoważoną, wielomianowo rozstrzygalną relacją binarną. 
    \textbf{Problem zliczania} związany z relacją $Q$ to następujące pytanie:
    Ile istnieje dla zadanego $x$ różnych $y$ takich, że $(x,y)\in Q$?
    (Wynik podajemy w postaci liczby binarnej.)
\end{definicja}

\begin{definicja}
    Klasa \textbf{\#P} zawiera wszystkie problemy zliczania związane z wielomianowo zrównoważonymi, wielomianowo rozstrzygalnymi relacjami.
\end{definicja}
% TODO: podać przykład z \#Matching

\begin{twierdzenie}
    $NP\subseteq \#P$
\end{twierdzenie}

\begin{definicja}
    Problem zliczający $A$ \textbf{redukuje się} do problemu $B$, jeśli istnieją funkcje $R,S\in L$ takie, że
    1) $x\in A \iff R(x)\in B$,
    2) $m$ jest odpowiedzią dla $R(x)$ w $B$ $\iff$ $S(m)$ jest odpowiedzią dla $x$ w $A$.
\end{definicja}

\begin{twierdzenie}
    $\#SAT$ jest $\#P$-zupełny. (wystarczy zmodyfikować tw. Cook'a.)
\end{twierdzenie}

\begin{twierdzenie}
    $\#HamiltonPath$ jest $\#P$-zupełny. (bo redukcja $3SAT$ do $HP$ była oszczędna.)
\end{twierdzenie}

\begin{twierdzenie}
    $Permanent$ jest $\#P$-zupełny.
\end{twierdzenie}

\begin{twierdzenie}
    $\#2SAT$ jest $\#P$-zupełny.
\end{twierdzenie}

\begin{definicja}
    Klasa $\oplus P$ zawiera wszystkie problemy zliczania w których pytamy o nieparzystość liczby rozwiązań.
\end{definicja}

\begin{twierdzenie}
    $\oplus SAT$ jest $\oplus P$-zupełny.
\end{twierdzenie}

\begin{lemat}
    $\oplus P$ jest zamknięta na dopełnienie, tj. $\oplus P = co-\oplus P$.
\end{lemat}



    \begin{twierdzenie}{(Algorytm alfa-beta)}
    2-os. EFG o sumie stałej. $minmax(v,lev)=$ return $alfabeta(v,lev,-\infty,\infty)$. Zwraca wartość gry dla równowagi doskonałej w grze o pełnej informacji.
\end{twierdzenie}
\begin{algorithm}[H]
    \SetAlgoLined
    \KwData{Input data if necessary}
    
    \If{$v\in L$}{
        return $u(v)$
    }
    \If{$P_V(v)=1$}{
        \For{$s\in succ(v)$}{
            $\alpha=\max(\alpha,Alfabeta(w,lev+1,\alpha,\beta))$\\
            \If{$\alpha\geq\beta$}{
                odcinamy gałąź Beta
            }
        }
        return $\alpha$
    }
    \If{$P_V(v)=2$}{
        \For{$s\in succ(v)$}{
            $\beta=\min(\beta,Alfabeta(w,lev+1,\alpha,\beta))$\\
            \If{$\alpha\geq\beta$}{
                odcinamy gałąź Alpha
            }
        }
        return $\beta$
    }
    \caption{Alfabeta(v,lev,$\alpha,\beta$)}
\end{algorithm}
\begin{definicja}{\textbf{Gra koalicyjna/kooperacyjna, z wypłatami ubocznymi}}
    $(P,v)$, $P$ sk. zb. graczy, $v:\mathcal{P}(P)\rightarrow\mathbb{R}$, $v(\emptyset)=0$, $v(S)$ zadaje 
    \textbf{siłę koalicyjną (wartość koalicyjną)} koalicji $S\subseteq P$, gdzie $v(S)$ ma być rozdystrybuowana między graczy w koalicji $S$.
    $P$ - \textbf{wielka koalicja}.
\end{definicja}
\begin{definicja}{\textbf{Gra kooperacyjna jest}}
    \textbf{nad-addytywna: } $(\forall S,T\subseteq P) ((S\cap T=\emptyset)\rightarrow(v(S)+v(T)\leq v(S\cup T)))$. Oznacza opłacalność łąćzenia się w większe koalicje.
    \textbf{monotoniczna: } $(\forall S\subseteq T\subseteq P) (v(S)\leq v(T))$.
    \textbf{wypukła: } $(\forall S,T\subseteq P) (v(S)+v(T)-v(S\cap T)\leq v(S\cup T))$. Gracze mają niemniejsze wypływy w większych koalicjach. Wypukła$\rightarrow$nad-addytywna.
\end{definicja}
\begin{fakt}\label{w10-f-normkoal}
    $(P=[n],S,u)$ gra w postaci normalnej. Wtedy $(P,v)$ grą koalicyjną, gdzie dla $T\subset P$: 
    $v(T)=\max\limits_{\pi_T\in\Delta_T^*}\min\limits_{\pi_{-T}\in\Delta_{-T}^*}\sum\limits_{i\in T}{u_i(\langle \pi_T,\pi_{-T} \rangle)}$.
\end{fakt}
\begin{fakt}
    Każdą grę nad-addytywną można otrzymać z pewnej gry w postaci normalnej za pomocą wzoru w Fakcie~\ref{w10-f-normkoal}.
\end{fakt}
\begin{definicja}
    $P=[n]$. Wektor wypłat w grze koalicyjnej to $x\in\mathbb{R}^n$, jest on:
    \textbf{racjonalny grupowo (alokacją/podziałem)} jeśli $\sum_{i=1}^n{x_i=v(N)}$;
    \textbf{racjonalny indywidualnie} jeśli $(\forall i\in [n]) (v(\{i\}) \leq x_i)$;
    \textbf{racjonalny koalicyjnie (stabilny)} jeśli $(\forall S\subseteq P)(v(S)\leq\sum_{i\in S}{x_i})$;
    \textbf{imputacją} jeśli jest indywidualnie racjonalną alokacją.
\end{definicja}
\begin{fakt}
    Nad-addytywna gra koalicyjna posada podział.
\end{fakt}
\begin{definicja}
    Zbiór $C(v)$ stabilnych podziałów to \textbf{rdzeń gry}. Może być pusty. Jest domknięty i wypukły.
\end{definicja}



    % \DeclareSymbolFont{symbolsC}{U}{txsyc}{m}{n}
% \DeclareMathSymbol{\notniFromTxfonts}{\mathrel}{symbolsC}{61}

\begin{definicja}
    Gracze $i,j\in P$ są \textbf{zamienni względem $v$} jeśli $(\forall S\subseteq P)((i,j\notin S)\rightarrow(v(S\cup\{i\})=v(S\cup\{j\})))$.
    Gracz $i$ jest \textbf{nieistotny względem $v$} gdy $(\forall S\not\owns i)(v(S)=v(S\cup\{i\}))$.
    Gracz $i$ ma \textbf{prawo veta} jeśli $v(P)\textbackslash \{i\}=0 \land v[\mathcal{P}(P)]\subset[0,\infty]$.
    Jeśli $v[\mathcal{P}(P)\subset\{0,1\}]$, to $(N,v)$ jest \textbf{grą prostą}; 
    równoważnie: gra koalicyjna jest prosta gdy $(\forall S\subseteq P)(v(S)\in\{0,1\})$.
\end{definicja}
\begin{fakt}
    Koalicyjna gra prosta ma niepusty rdzeń IFF ma graczy z prawem veta. Składa się on wtedy z podziałów, gdzie gracze z prawem veta odstają sumaryczną wypłatę 1.
\end{fakt}
\begin{definicja}{\textbf{Rozwiązanie gry}}
    Dla gry $(P,v)$ pewien zbiór podziałów $\Phi(v)$ zależny od wyboru $v$. 
    Jeśli rozwiązaie jest 1-el., to $\Phi(v)$ nazywamy \textbf{wartością gry}, a $\Phi_i(v)$ wartością $i$-tego gracza.
    % Czasami $\Phi$ jest postrzegane jako multifunkcja 
\end{definicja}
\begin{definicja}{\textbf{Wartość Shapleya (SV)}}
    Wartość gry dla $i\in P$: $\Phi_i(v)=\sum\limits_{S\subseteq P\textbackslash\{i\}}{\frac{|S|!(|P\textbackslash S|-1)!}{|P|!}(v(S\cup\{i\})-v(S))}$
    Mówi jaki jest średni wkład w siłę koalicyjną gracza $i$ po wszystkich możliwych liniowych porządkach dodawania pojedynczych graczy do koalicji.
\end{definicja}
\begin{fakt}
    W grach wypukłych, wartość Shapleya jest w rdzeniu gry.
\end{fakt}
\begin{fakt}{\textbf{Własności SV}}
    \textbf{Wydajność:} SV jest podziałem.
    \textbf{Symetria:} Dla wszystkich $i,j$ zamiennych względem $v$ mamy $\Phi_i(v)=\Phi_j(v)$.
    \textbf{Liniowość:} $\Phi_i(av+bw)=a\Phi_i(v)+b\Phi_i(w)$.
    \textbf{Nieistotny gracz:} Jeśli $i$ nieistotny względem $v$, to $\Phi_i(v)=0$.
    \textbf{Addytywność:} Gdy $v$ jest nad-addytywna, to $\Phi(v)$ indywiidualnie racjonalna; 
    jeśli $v$ pod-addytywna, to $(\forall i) \Phi_i(v)\leq v(\{i\})$;
    jeśli addytywna, zachodzi równość. \\
    Zbiór pewnych cech, których używamy do predykcji możemy uznać za graczy. Siły koalicji będą
wskazywały na jakość predykcji. Wtedy wartość Shapleya można użyć w kontekście analizy wariancji ANOVA, czy w machine
learningu (np. LIME) Podobny model można wykorzystać w szeregach czasowych do zdefiniowania slidShap (X 2023), gdzie
siły koalicyjne są wyznaczane przy pomocy entropii Shannona, a który to służy do predykcji przyszłych wartości ciągu
wektorów losowych.
\end{fakt}
\begin{twierdzenie}{(Bondareva-Shapley)}
    Dla gry koalicyjnej, posiadanie niepustego rdzeń jest równoważne z warunkiem: 
    $(\forall i\in P) (\forall \alpha:\mathcal{P}(P)\rightarrow[0,1]) \left(\sum\limits_{S\owns i}{\alpha(S)=1} \right)\rightarrow \left(\sum\limits_S{\alpha(S)v(S)\leq v(P)} \right)$.
\end{twierdzenie}
\begin{definicja}{(\textbf{Indeksy siły Shapleya-Shubika i Banzhafa})}
    Rozważmy prostą grę. $v(S)=1$ oznacza, że można uformować koalicję $S$, która będzie rządzić (wygrywać wybory). 
    Zakładamy że $v(P)=1$, $v(\emptyset)=0$, $v$ niemalejąca, dla $S\subseteq P$ jesli $v(S)=1$ to $v(P\textbackslash S)=0$.
    \textbf{Indeksem siły Shapleya-Shubika} $i$-tego gracza $Sh(i)$ jest wartość Shapleya $i$-tego gracza w takiej grze.
    Niech $S\subseteq P$. Określamy $K(S)=\{i\in S: v(S)=1, v(S\textbackslash\{i\})=0\}$ - zbiór kluczowych koalicjantów w koalicji $S$.
    \textbf{Indeksem Banzhafa} gracza $i$ nazywamy: $B(i)=\frac{|\{S\subseteq P: i\in K(S)\}|}{\sum\limits_{j\in P}|\{S\subseteq P: j\in K(S)\}|}$.
    W grach prostych $\sum\limits_{i\in P}{Sh(i)} = \sum\limits_{i\in P}{B(i)} = 1$.
\end{definicja}
% TODO: DODAĆ UWAGĘ 79 (PARADOKS DE CONDORCETA)



    \input{wyklady/wyklad12}
    \input{wyklady/wyklad13}
    \input{wyklady/wyklad14}
    \input{wyklady/wyklad15}
\end{multicols*}

\end{document}