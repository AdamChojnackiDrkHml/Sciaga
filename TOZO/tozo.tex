%%%% Uniwersalny Szablon Infy WPPT v4.2 (02.02.2018) %%%%

% uklad dokumentu
\documentclass[8pt]{extarticle}
\usepackage{xparse}
\usepackage[margin=0pt]{geometry}
\usepackage{enumerate} 
\frenchspacing
\linespread{0.5}
 \setlength{\parindent}{0pt}

% pakiety matematyczne
\usepackage{amssymb}   
\usepackage{amsthm}
\usepackage{amsmath}
\usepackage{amsfonts}
\usepackage{mathabx}

\usepackage{tikz}
\def\thm@space@setup{\thm@preskip=0pt
\thm@postskip=0pt}
% jezyk polski
\usepackage[polish]{babel}
\usepackage[utf8]{inputenc}
\usepackage{polski}
\usepackage[T1]{fontenc}

% hiperlacza
\usepackage{hyperref}
\hypersetup{
    colorlinks,
    citecolor=black,
    filecolor=black,
    linkcolor=black,
    urlcolor=black
}

\newtheoremstyle{mystyle}
  {1pt}%   Space above
  {1pt}% 
  {\slshape}%  Body font
  {}%          Indent amount (empty = no indent, \parindent = para indent)
  {\bfseries}% Thm head font
  {.}%         Punctuation after thm head
  {0.5em}%     Space after thm head: " " = normal interword space;
     %         \newline = linebreak
  {}%          Thm head spec (can be left empty, meaning `normal')

% wstawianie zdjec
\usepackage{graphicx} 

% deklaracja zadania
\theoremstyle{mystyle}
\newtheorem{zadanie}{Zadanie}[subsection]


% deklaracja metody
% \theoremstyle{remark}
\newtheorem*{metoda}{Metoda}
% \theoremstyle{plain}

% deklaracja rozwiazania
% \theoremstyle{remark}
\newtheorem*{rozwiazanie}{Rozwiązanie}
% \theoremstyle{plain}

\newtheorem{definicja}{Def}
\newtheorem{twierdzenie}{Tw}
\newtheorem{fakt}{F}

% zestaw - mpis
\newcommand{\EE}{\mathbb{E}}
\newcommand{\PP}{\mathbb{P}}
\newcommand{\Var}{\mathrm{Var}}
\newcommand{\DD}{\mathbb{D}}
\newcommand{\Cal}{\mathcal}
\newcommand{\bb}{\mathbb}
\newcommand{\red}[1]{\textcolor{red}{#1}}

% zestaw - programowanie
\usepackage{listings} 
\usepackage{minibox}
\usepackage{minted}
\usemintedstyle{borland}

% zestaw - akiso
%\usepackage{karnaugh-map} %%% nie ma w bazie overleaf
\usepackage{karnaughmap}
\usepackage{circuitikz}
\usepackage[inline]{enumitem}
\usepackage{tikz}
\usetikzlibrary{automata,positioning}

% zestaw - dyskretna
\newcommand{\stirf}[2]{\genfrac{[}{]}{0pt}{}{#1}{#2}}
\newcommand{\stirs}[2]{\genfrac{\{}{\}}{0pt}{}{#1}{#2}}

% zestaw - jftt
\newcommand{\oor}{\mathop{|}}
\DeclareMathOperator{\Lead}{Leading}
\DeclareMathOperator{\Trail}{Trailing}
\DeclareMathOperator{\fst}{First}
\DeclareMathOperator{\fol}{Follow}



\usepackage{multicol}
\usepackage{multirow}
\usepackage{float}
\setlength{\columnseprule}{1pt}
\def\columnseprulecolor{\color{black}}
\begin{document}

\begin{multicols*}{4}
    \begin{definicja}{Przestrzeń topologiczna}
    Para $(X, \Cal{O})$, gdzie $X$ jest zbiorem, a $\Cal{O} \subset \Cal{P}(X)$ (\textbf{topologia}):
    \begin{enumerate*}[label=\roman*]
        \item $\emptyset, X \in \Cal{O}$,
        \item $U, V \in \Cal{O} \Rightarrow U \cap V \in \Cal{O}$,
        \item $\Cal{T} \subset \Cal{O} \Rightarrow \bigcup\Cal{T} \in \Cal{O}$
    \end{enumerate*}
    Elementy $\Cal{O}$ nazywamy \textbf{zbiorami otwartymi}. Ich dopełnienia w $X$ nazywamy \textbf{zbiorami domkniętymi}.
\end{definicja}

\begin{definicja}
    W przestrzeni topologicznej $(X, \Cal{O})$ mówimy, że funkcja $f$ jest ciągła $\Leftrightarrow$ $\forall_{(\Cal{U} \in \Cal{O})}(f^{-1}[\Cal{U}] \in \Cal{O})$.
\end{definicja}

\begin{definicja}
    Ciąg ${(x_{n})}_{n}$ w przestrzeni 
    metrycznej 
    $(X, d)$ nazywamy \textbf{ciągiem Cauchy'ego}, 
    gdy:
    $((\forall \varepsilon > 0)(\exists N_0)(\forall n, m > N_0) d(x_n, x_m) < \varepsilon)$
    Mówimy, że prz.m. jest \textbf{zupełna}, jeśli każdy ciąg Cauchy'ego jest zbieżny.
\end{definicja}

\begin{twierdzenie}{Banach, o punkcie stałym, kontrakcji}
    Niech $(X, d)$ zupełna prz.m., $f: X \rightarrow X$ będzie \textbf{kontrakcją/odwzorowaniem zwężającym} ze stałą $\alpha < 1$.
    $(\forall x, y \in X)d(f(x), f(y)) < \alpha\cdot d(x, y)$. Wtedy $f$ ma dokładnije jeden punkt stały $x^*$, $f(x^*)=x^*$.
    Jeśli $x_0 \in X$ jest dowolnym elementem $X$, to ciąg $x_{n+1} = f(x_n)$, zbiega do $x^*$.
\end{twierdzenie}

\begin{definicja}
    Podzbiór przestrzeni topologicznej $(X, \Cal{D})$ nazywamy \textbf{zwartym}, jeśli z dowolnego pokrycia ${\{U_t\}}_{t\in T}$
    $X$ zbiorami otwartmi, można wybrać podpokrycie ${\{U_{t_i}\}}_{i = 1}^{n}$ skończone.
\end{definicja}

\begin{fakt}
    W przestrzeni $\bb{R}^k$ z naturalną topologią, zwartość podprzestrzeni jest równowazna z jej domkniętością i ograniczoniością.
\end{fakt}

\begin{fakt}
    W prz.m. zwartość zbioru $K$ jest równoznaczna z tym, że dla każdego ciągu w $K$, istnieje podciąg zbieżny do pewnego punktu $x \in K$.
\end{fakt}

\begin{fakt}
    W prz.m. każdy zbiór zwarty jest zupełny.
\end{fakt}

\begin{twierdzenie}
    Niech $K$, będzie zbiorem zwartym. Każda funkcja ciągła $f: K \rightarrow R$ osiąga swoje kresy (maksima i minima o ile istnieją).
\end{twierdzenie}

\begin{definicja}
    W przestrzeni liniowej $\bb{V}$ nad ciałem $\bb{R}$, niech $n \in \bb{N}, x_i \in \bb{V}$ i $\lambda_i \in K$, dla $i \in [n]$,
    będą takie, że $\sum\limits_{i=1}^{n}\lambda_i = 1$, gdzie $(\forall i \in [n])(\lambda_i \int [0, 1])$. Mówimy wtedy, że punkt postaci
    $\sum\limits_{i=0}^{n} \lambda_i x_i \in \bb{V}$ jest \textbf{kobinacją wypukłą} punktów ${(x_i)}^{n}_{i=1}$ z wagami ${(\lambda_i)}^{n}_{i=1}$.
\end{definicja}

\begin{definicja}
    Podzbiór prz. l. $X \subset \bb{V}$ jest \textbf{wypukły}, jeśli kombinacja wypukład dwóch doowlnych punktów z $X$ jest elementem $X$.
    W przeciwnym razie, jest \textbf{wklęsły}.
\end{definicja}

\begin{definicja}
    $X \subset \bb{V}$. $f: X \rightarrow \bb{R}$ jest \textbf{wypukła} gdy: 
    $(\forall x, y \in X)(\forall \lambda \in [0, 1])((\lambda x + (1 - \lambda)y \in X) \Rightarrow ((\lambda f(x) + (1 - \lambda)f(y)) \geq f(\lambda x + ( 1 - \lambda)y)))$
\end{definicja}

\begin{definicja}
    Zbiór wektorów ${(x_i)}^{k}_{i=0}$ w prz.l. $\bb{V}$ jest \textbf{afnicznie niezależny wzlędem} $x_0$ jeśli ${(x_i - x_0)}_{i=0}^{k}$ 
    jest liniowo neizależny od $\bb{V}$.
\end{definicja}

\begin{definicja}
    \textbf{Sympleksem k-wymiarowym} nazywamy najmniejszy możliwy podzbiór prz.l. $\bb{V}$ nad ciałem $\bb{R}$ taki,
    że zawiera wszystkie możliwe kombinacje wypukłe pewnych elem. ${(x_i)}_{i=0}^{k} \subset \bb{V}$,
     afnicznie niezależny wzgl. $x_0$.
    Elementy te nazywamy \textbf{wierzchołkami} sympleksu.
\end{definicja}

\begin{twierdzenie}
    Jeśli $S$ jest sympleksem, a $f: S \rightarrow S$ jest odwzorowaniem ciągłym, to $f$ ma punkt stały.
\end{twierdzenie}

\begin{definicja}
    W prz. m. $(X, d)$, \textbf{średnicą zbioru} $A \subset X$ nazywamy $diam(A) := \sup{\{d(x,y)\;:\; x, y \in A\}}$
\end{definicja}

\begin{definicja}
    Mówimy, że $f: X \rightarrow \bb{R}$ ma własność \textbf{wykresu domkniętego}, jeśli dla ciągu ${(x_i)_{i=1}^{\infty}}$, 
    zbieżnego do $x^*$, dowolny ciąg ${(y_i)_{i=1}^{\infty}}$, zbieżny do $y^*$, spełnienie warunku $(\forall i \in \bb{N})(y_i \in f(x_i))$,
    pociąga za sobą, że $y^* \in f(x^*)$.
\end{definicja}

\begin{twierdzenie}
    Niech $S$ będzie sympleksem $k$-wymiarowym, a $f: S \rightarrow \cal{P}(S)$ będzie miał własność wykresu domkniętego
    oraz spełnia warunek $(\forall s \in S)(f(s) \text{ jest zwarty i wypukły})$.
    Istnieje wtedy $x^* \in S$, że $x^* \in f(x^*)$.
\end{twierdzenie}
    \begin{definicja}
    Niech $P=\{p_1, p_2, \dots, p_n\}$ będzie zbiorem $n$ graczy. Każdy z graczy $p_i, i \in [n]$,
    ma swój zwarty zbiór strategii (\textbf{czystych}) $S_i$.
    W każdej realizacji gry, każdy z graczy $p_i$ wybiera swoją strategię $s_i \in S_i$,
    tworząc w ten sposób wektor strategii $s = (s_1, s_2, \dots, s_n)$.
    Wtedy $S = \bigtimes_{n=1}^{n} S_i$ to zbiór możliwych \textbf{profili strategii}(wektorów strategii). 
    Każdy profil strategii $s$ jest jednznacznie wyznacza wektor wypłat $u(s) = (u_1(s), \dots, u_n(s))$ ($u: S \rightarrow V^n$, gdzie $V$ jest prz.l.),
    czyli zysków (lub strat) poszczególnych graczy. Każda z $u_i$ jest nazywana \textbf{funkcją wypłat} dla gracza $p_i$. 
    $(P, S, u)$ nazywamy \textbf{grą w postaci normalnej}. Jeśli $|A| < \aleph_0$, to mówimy, że gra jest skończona.
\end{definicja}

\begin{definicja}
    Niech $P = [n]$. Dla danego profilu strategii $s \in S$ określamy $s_{-i}$,
    jako profil strategii $(s_1, \dots, s_{i-1}, s_{i+1}, \dots, s_n)$ z pominięciem strategii czystej $i$-tego gracza.
    Będziemy wtedy zapisywać skrótowo $s = \langle s_i, s_{-i} \rangle$.
    Mówimy, że statega czysta $s_{i}^{*}$:
    \begin{enumerate*}[label=\roman*)]
        \item \textbf{słabo dominuje} $s_{i}'$ jeśli $(\forall s \in S)(u_i(\langle s_{i}^{*}, s_{-i} \rangle) \geq u_i(\langle s_{i}^{'}, s_{-i} \rangle))$
        \item \textbf{silnie dominuje} $s_{i}'$ jeśli $(\forall s \in S)(u_i(\langle s_{i}^{*}, s_{-i} \rangle) > u_i(\langle s_{i}^{'}, s_{-i} \rangle))$
        \item \textbf{najlepszą odpowiedzią na} $s_{-i}$ $(\forall s_i \in S_i)(u_i(\langle s_{i}^{*}, s_{-i} \rangle) \geq u_i(\langle s_{i}, s_{-i} \rangle))$
    \end{enumerate*}
    Mówimy, że profil strategii $s^{*} = (s_{1}^{*}, \dots, s_{n}^{*}) \in S$:
    \begin{enumerate*}[label=\roman*)]
        \item jest \textbf{strategią słabo dominującą}: $(\forall i \in [n])(\forall s \in S)(u_i(\langle s_{i}^{*}, s_{-i} \rangle) \geq u_i(s))$
        \item jest \textbf{strategią silnie dominującą}: $(\forall i \in [n])(\forall s \in S, s_i \neq s_{i}^{*})(u_i(\langle s_{i}^{*}, s_{-i} \rangle)$ $> u_i(s))$
        \item jest \textbf{równowagą Nasha w strategiach czystych}: $(\forall i \in [n]) (s_{i}^{*}$ jest najlepszą odpowiedzią na $s_{-i}^{*})$.
        \item jest \textbf{optimum Pareta}: $(\forall s \in S\setminus{s^{*}})(\{[(\exists j \in [n])u_{j}(s) < u_{j}(s^{*})]$ $\lor [u(s) = u(s^{*})]\})$.
    \end{enumerate*}
    Zbiór wszystkich optimum Pareta, jest nazywany \textbf{frontem Pareta}.
\end{definicja}
\begin{fakt}
    Dla każdego gracza, słaba dominacja w sensie strategii czystych jest relacją częsciowego porządku.
\end{fakt}
\begin{fakt}
    $s \in S$ jest rowiązaniem (słabym) $\Leftrightarrow$ $(\forall p_i \in P)$ ($s_i$ słabo dominuje wszystkie inne strategie tego gracza).
\end{fakt}
\begin{fakt}
    Jeśli $s \in S$ jest rozwiązaniem słabym, to dla każdego gracza $p_i$, $s_i$ jest najlepszą odpowiedzią,
    na każdy możliwy profil strategii innych graczy $s_{-i}'$. W szczególności jest równowagą Nasha dla strategii czystych.
\end{fakt}
\begin{fakt}
    Jeśli istnieją strategie czyste ściśle zdominowane, usuwaj je, dopóki się da. 
    W wyniku takiego algorytmu otrzymamy uproszczoną grę.
    Nawet gdy dojdziemy do gry z jedną strategią, to nie musi być to strategia dominująca nawet w słabym sensie.
\end{fakt}
\begin{twierdzenie}
    Jeśli istnieje rozwiązanie ściśle dominujące, to jest ono jedno i algorytm eliminacji prowadzi do tego rozwiązania
\end{twierdzenie}
\begin{definicja}
    \textbf{Gra koordynacyjna} to gra, w której gracz ma identyczny zbiór strategii czystych oraz wypłata gracza
    $p \in P$ jest niemalejącą funkcją ze względu na rosnącą liczbę graczy, która wybierze tę samą czystą strategię. 
    \textbf{Gra anty-koordynacyjna} analogicznie ma nierosnącą funkcję.
    \textbf{Gra dysokoordynacyjna} dla dwóch osób, jeden z graczy dąży do koordynacji, drugi  do antykoordynacji strategii czystych.
\end{definicja}
    \begin{definicja}
    \textbf{Problemem decyzyjnym} nazywamy rozstrzygnięcie pewnej własności dla języka danych. Zakładamy, że rozstrzygnięcie należenia do języka danych jest prostsze niż rozstrzygnięcie posiadania tej własności. Przykład danych nazywamy pozytywnym, jeśli posiada tę własność, negatywnym, jeśli jej nie posiada. Dopełnieniem problemu jest pytanie o przeciwną własność.
\end{definicja}

\begin{definicja}
    Niech $f$ będzie funkcją $\mathbb{N} \rightarrow \mathbb{N}$. $f$ jest \textbf{właściwą funkcją złożoności}, jeżeli jest niemalejąca i istnieje maszyna Turinga $M_f$ taka, że dla każdego $n \in \mathbb{N}$ i dla każdego $x$ długości $n$ zachodzi $M_f(x) = \sqcap^{f(|x|)}$ oraz $M_f$ pracuje w czasie $O(|x| + f(|x|))$ i pamięci $O(f(|x|))$.
\end{definicja}

\begin{definicja}
    Maszyna Turinga M jest \textbf{dokładna}, jeżeli istnieją takie właściwe funkcje złożoności $f$ i $g$, że dla każdego $x$ długości $n$ obliczenie $M(x)$ zatrzymuje się dokładnie po $f(n)$ krokach i wszystkie jej taśmy robocze zawierają dokładnie $g(n)$ symboli (innych niż $\sqcup$). Zakładamy dodatkowo, że taśma wejściowa jest tylko do odczytu.
\end{definicja}

\begin{lemat}
    Maszyny Turinga z ograniczoną pamięcią zawsze się zatrzymują.
\end{lemat}

\begin{definicja}
    \begin{itemize*}[label={}]
        \item $TIME(f)$ --- zbiór wszystkich dokładnych, deterministycznych maszyn Turinga pracujących w czasie co najwyżej $f$,
        \item $NTIME(f)$ --- zbiór wszystkich dokładnych, niedeterministycznych maszyn Turinga pracujących w czasie co najwyżej $f$,
        \item $SPACE(f)$ --- zbiór wszystkich dokładnych, deterministycznych maszyn Turinga pracujących z pamięcią co najwyżej $f$
        \item $NSPACE(f)$ --- zbiór wszystkich dokładnych, niedeterministycznych maszyn Turinga pracujących z pamięcią co najwyżej $f$
    \end{itemize*}
\end{definicja}

\begin{definicja}
    \textbf{Podstawowe klasy złożoności obliczeniowej}:
    \begin{itemize*}[label={}]
        \item $L = SPACE(\log n)$,
        \item $NL = NSPACE(\log n)$,
        \item $P = \bigcup_{j \in \mathbb{N}} TIME(n^j)$,
        \item $NP = \bigcup_{j \in \mathbb{N}} NTIME(n^j)$
        \item $PSPACE = \bigcup_{j \in \mathbb{N}} SPACE(n^j)$
        \item $NPSPACE = \bigcup_{j \in \mathbb{N}} NSPACE(n^j)$
        \item $EXP = \bigcup_{j \in \mathbb{N}} TIME(2^{n^j})$
    \end{itemize*}
\end{definicja}

\begin{definicja}
    Mówimy, że problem $A$ należy do klasy złożoności $\mathcal{C}$, jeśli istnieje
    maszyna Turinga $M \in \mathcal{C}$ rozstrzygająca $A$.
\end{definicja}

\begin{twierdzenie}
    Następujące zawierania są prawdziwe:
    \begin{itemize*}[label={}]
        \item $TIME(f) \subseteq NTIME(f)$ oraz $SPACE(f)\subseteq NSPACE(f)$,
        \item $NTIME(f) \subseteq SPACE(f)$,
        \item $NSPACE(f) \subseteq TIME(k^{\log + f})$.
    \end{itemize*}
    Wniosek: $L \subseteq NL \subseteq P \subseteq NP \subseteq PSPACE \subseteq EXP$.
\end{twierdzenie}

\begin{twierdzenie}
    \textbf{Savitch} --- $REACHABILITY$, problem sprawdzenia, czy w grafie $G$ istnieje ścieżka między dwoma ustalonymi wierzchołkami $a$ i $b$, należy do $SPACE((\log n)^2)$. Wniosek: $NSPACE(f(n)) \subseteq SPACE((f(n))^2)$, dla $f \in \Omega(\log)$. $PSPACE = NPSPACE$.
\end{twierdzenie}

\begin{definicja}
    $H_f = \{ M; x : M$ akceptuje $x$ w co najwyżej $f(|x|)$ krokach $\}$.
\end{definicja}

\begin{lemat}
    $H_f \in TIME((f(n))^3)$.
\end{lemat}

\begin{lemat}
    $H_f \notin TIME(f(\lfloor \frac{n}{2} \rfloor))$.
\end{lemat}

\begin{twierdzenie}
    \textbf{O hierarchii czasowej} --- Jeżeli $f(n) \geq n$, to $TIME(f(n)) \subsetneq TIME((f(2n + 1))^3)$. Wniosek: $P \subsetneq EXP$.
\end{twierdzenie}

\begin{twierdzenie}
    \textbf{O hierarchii pamięciowej} --- $SPACE(f(n)) \subsetneq SPACE(f(n) \log f(n))$. Wniosek: $L \subsetneq PSPACE$.
\end{twierdzenie}

\begin{twierdzenie}
    Jeśli $f(n) \in o(\log \log n)$ to $SPACE(f) = SPACE(0)$. $SPACE(0)$ to języki regularne.
\end{twierdzenie}

\begin{definicja}
    \textbf{Redukcja} --- Problem $A$ jest co najmniej tak trudny, jak $B$, jeżeli istnieje funkcja $R \in L$ taka, że $x$ jest pozytywnym przykładem $B \iff R(x)$ jest pozytywnym przykładem $A$. Mówimy, że B redukuje się do A. (Redukcja należy do $L$, najsłabszej klasy, tak aby sama redukcja nie mogła rozwiązać problemu).
\end{definicja}

\begin{lemat}
    Jeśli $R_1$ i $R_2$ są redukcjami to złożenie $R_3 = R_2 \circ R_1$ też jest redukcją. 
\end{lemat}

\begin{definicja}
    Problem $A$ jest trudny dla klasy $\mathcal{C}$ ($\mathcal{C}$-trudny), jeżeli każdy problem $B \in \mathcal{C}$ redukuje się do $A$.
\end{definicja}

\begin{definicja}
    Problem $A$ jest zupełny dla klasy $\mathcal{C}$ ($\mathcal{C}$-zupełny), jeśli $A \in \mathcal{C}$ i $A$ jest $\mathcal{C}$-trudny.
\end{definicja}

\begin{lemat}
    Jeśli $A$ jest $\mathcal{C}$-trudny i $A$ redukuje się do $B$ to $B$ jest $\mathcal{C}$-trudny.
\end{lemat}

\begin{twierdzenie}
    $REACHABILITY$ jest $NL$-zupełny.
\end{twierdzenie}

\begin{definicja}
    $CIRCUIT$ $VALUE$ to problem, czy dla danej sieci boolowskiej $C(x_1,\dots,x_n)$ i danego wartościowania wejść $x_1,\dots,x_n$ wartością wyjściową sieci będzie $1$.
\end{definicja}

\begin{lemat}
    $CIRCUIT$ $VALUE \in P$.
\end{lemat}

\begin{twierdzenie}
    $CIRCUIT$ $VALUE$ jest $P$-trudny. Wniosek: $CIRCUIT$ $VALUE$ jest $P$-zupełny.
\end{twierdzenie}

\begin{definicja}
    $CIRCUIT$ $SATISFABILITY$ to problem, czy dla danej sieci boolowskiej $C(x_1,\dots,x_n)$ istnieje wartościowanie wejść $x_1,\dots,x_n$ takie, że wartością wyjściową sieci będzie $1$ (sieć jest spełnialna).
\end{definicja}

\begin{lemat}
    $CIRCUIT$ $SATISFABILITY \in NP$.
\end{lemat}

\begin{twierdzenie}
    $CIRCUIT$ $SATISFABILITY$ jest $NP$-trudny. Wniosek: $CIRCUIT$ $SATISFABILITY$ jest $NP$-zupełny.
\end{twierdzenie}

\begin{definicja}
    $3SAT$ to problem, czy formuła w postaci koniunkcji klauzul zawierających co najwyżej alternatywę 3 literałów będących nazwą zmiennej lub jej negacją, jest spełnialna. W wersji z dokładnie 3 literałami moglibyśmy formułę zapisać następująco: $\phi(x_1,\dots,x_n) = \bigwedge_{i = 1}^{m} (l_{i1} \lor l_{i2} \lor l_{i3})$, gdzie $l_{ij} \in \{x_1,\dots,x_n, \neg x_1,\dots, \neg x_n\}$.
\end{definicja}

\begin{twierdzenie}
    $3SAT$ jest $NP$-zupełny.
\end{twierdzenie}

\begin{definicja}
    $INDEPENDENT$ $SET$ to problem, czy dla danego grafu $G = (V, E)$ i $K \in \mathbb{N}$, istnieje $I \subseteq V$ o mocy $K$ taki, że $I$ jest niezależny, czyli nie istnieje krawędź w $E$ łącząca dwa wierzchołki z $I$.
\end{definicja}

\begin{twierdzenie}
    $INDEPENDENT$ $SET$ jest $NP$-zupełny.
\end{twierdzenie}

\begin{definicja}
    Algorytm nazywamy \textbf{pseudowielomianowym}, jeśli zależy wielomianowo od liczb występujących w danych, a nie jest wielomianowy od długości ich zapisu.
\end{definicja}

\begin{definicja}
    Jeżeli problem pozostaje $NP$-zupełny nawet wtedy, gdy wymagamy by dowolny przykład długości $n$ zawierał liczby wielkości co najwyżej $p(n)$, dla pewnego wielomianu $p$, to problem nazywamy \textbf{silnie} $NP$-zupełnym.
\end{definicja}

\begin{definicja}
    $co$-$NP$ --- klasa problemów wielomianowych z łatwymi dowodami negatywnymi (zmiana definicji akceptowania: akceptujemy kiedy wszystkie ścieżki obliczeń kończą się stanem TAK, odrzucamy jeśli choć jedna kończy się stanem NIE).
\end{definicja}

\begin{lemat}
    $P \subseteq NP \cap co$-$NP$
\end{lemat}

\begin{lemat}
    Jeżeli problem $A$ jest $NP$-zupełny, to jego dopełnienie jest $co$-$NP$-zupełne.
\end{lemat}

\begin{lemat}
    Jeżeli problem $co$-$NP$-zupełny jest w $NP$, to $NP = co$-$NP$. Problemy otwarte: Czy $NP = co$-$NP$? Czy $P = NP \cap co$-$NP$?
\end{lemat}

\begin{lemat}
    Problem optymalizacyjny, którego wersja decyzyjna jest $NP$-trudna, jest również $NP$-trudny.
\end{lemat}
    \begin{definicja}
    \textbf{Metoda Williamsa} (chłopski rozum): szukamy najlepszej strategii zrandomizowanej. Jeśli strategia jest zdominowana, to jej p-p ustawiamy na 0. Pozostałe w zależności od p-p, że będą najlepsze i ich wartości wypłat. Graficznie: 
    \begin{enumerate*}[label=\roman*)] 
        \item  Przedstaw strategie 2. gracza jako punkty w przestrzeni $X \times Y$ odpowiedzi gracza 1. 
        \item Znajdź otoczkę wypukłą tych punktów. 
        \item Poprowadź prostą X = Y . Znajdź punkt z otoczki wypukłej, przeciętej z tą prostą, który minimalizuje wypłatę pierwszego gracza.
    \end{enumerate*}
\end{definicja}

\begin{twierdzenie}
    (Nash). Rozważmy $\Gamma = (P = [n],S,u)$ będzie grą w postaci normalnej. Niech każdy ze zbiorów strategii graczy $S_i$ będzie zwartym i wypukłym podzbiorem $\mathbb{R}^k$. Dla każdego $i \in [n]$, niech $u_i$ będzie ciągła oraz $(\forall s_{-i} \in S) s_i \mapsto u_i(\langle s_i, s_{-i} \rangle)$ - wklęskła. Wtedy $\Gamma$ posiada co najmniej jedną mieszaną równowagę Nasha.
\end{twierdzenie}
    \begin{lemat}
    Niech $G = (V, E)$ i $(x, y) \in E$. Wtedy
    $vc(G) = \min \{vc(G \setminus x), vc(G \setminus y)\} + 1$
    gdzie $vc(G)$ to minimalny rozmiar pokrycia wierzchołkowego grafu $G$.
\end{lemat}

\begin{lemat}
    Dla $G = (V, E)$ jeśli $vc(G) \leq k$, to $|E| \leq k(|V| - 1)$.
\end{lemat}

\begin{definicja}
    Niech $\mathbb{G}$ będzie zbiorem wszystkich grafów. \textbf{Parametrem grafu}
    nazywamy dowolną funkcję $p : \mathbb{G} \rightarrow \mathbb{N}$.
\end{definicja}

\begin{definicja}
   \textbf{Problemem parametrycznym} nazywamy parę $(\Pi, p)$, gdzie $\Pi$ --- problem, a $p$ - parametr.
\end{definicja}

\begin{definicja}
    Algorytm $A$ rozwiązujący $(\Pi, p)$ nazywamy \textbf{Fixed Parameter Tractable} (FPT),
    jeśli istnieje obliczalna funkcja $f : \mathbb{N} \rightarrow \mathbb{N}$ t. że złożoność
    algorytmu $A$ może być wyrażona jako $\mathcal{O}(f(p(G))|G|^{\mathcal{O}(1)})$.
\end{definicja}

\begin{definicja}
    Problem parametryczny $(\Pi, p)$ jest FPT, jeśli istnieje algorytm FPT,
    parametryzowany przez $p$, który go rozwiązuje.
\end{definicja}

\begin{definicja}
    Mając problem parametryczny $(\Pi, p)$, \textbf{algorytm kernelizacji} zastępuje dane
    $(I, p(I))$ danymi (zredukowanymi) $(I', p'(I'))$ (zwanymi \textbf{kernelem} problemu) t. że
    \begin{enumerate*}[label=\roman*)]
        \item $p'(I') \leq g(p(I))$ i $|I'| \leq f(p(i))$ dla pewnych obliczalnych funkcji $f, g$ zależnych
        wyłącznie od $p(I)$ (nie od $|I|$).
        \item $(I, p(I)) \in \Pi$ wtedy i tylko wtedy, gdy $(I', p'(I')) \in \Pi$
        \item redukcja z $(I, p(I))$ do $(I', p'(I'))$ jest obliczalna w czasie wielomianowym od $|I| + p(I)$
    \end{enumerate*}
\end{definicja}

\begin{lemat}
    Niech $G = (V, E)$ i $v \in V$ ma stopień większy niż $k$. Wtedy $v$ należy do każdego pokrycia wierzchołkowego
    rozmiaru co najwyżej $k$.
\end{lemat}

\begin{lemat}
    Niech $G = (V, E)$. Jeśli $vc(G) \leq k$ i żaden wierzchołek w $G$ nie ma stopnia większego niż $k$, to $|E| \leq k^2$.
\end{lemat}

\begin{lemat}
    Jeśli $G$ ma pokrycie wierzchołkowe o rozmiarze co najwyżej $k$,
    to \textbf{Algorytm kompresji dla $VertexCover(k)$} zwróci pokrycie wierzchołkowe $G$
    rozmiaru co najwyżej $k$ w czasie $\mathcal{O}(|E|2^{|Q|})$.
\end{lemat}

\begin{lemat}
    Jeśli $G$ ma pokrycie wierzchołkowe rozmiaru co najwyżej $k$,
    to \textbf{Algorytm FPT z iteracyjną kompresją dla $VertexCover(k)$} zwróci
    pokrycie wierzchołkowe rozmiaru co najwyżej $k$ w czasie $\mathcal{O}(|E||V|2^{k + 1})$.
\end{lemat}
    \begin{definicja}
    Wielomianowa dokładna (niedeterministyczna) maszyna Turinga jest typu
    \textbf{Monte Carlo} dla problemu $A$, jeśli dla każdego $x \in A$ co najmniej
    połowa obliczeń na słowie $x$ jest akceptująca, a dla $x \notin A$
    wszystkie obliczenia odrzucają $x$.
\end{definicja}

\begin{definicja}
    \textbf{Klasa RP} (randomized polynomial time) to zbiór problemów posiadających
    maszyny Turinga typu Monte Carlo.
\end{definicja}

\begin{lemat}
    Jeżeli dokładna wielomianowa maszyna Turina $M$ dla każdego pozytywnego przykładu $x$
    ma co najmniej $0 < \varepsilon < \frac{1}{2}$ obliczeń akceptujących, to $M$ jest typu Monte Carlo.
\end{lemat}

\begin{lemat}
    $P \subseteq RP \subseteq NP$
\end{lemat}

\begin{definicja}
    \textbf{Klasa coRP} to zbiór problemów posiadających wielomianową dokładną (niedeterministyczną)
    maszynę Turinga taką, że dla każdego negatywnego przykładu $x$ co najmniej połowa obliczeń odrzuci $x$,
    a dla pozytywnego $x$ wszystkie obliczenia zaakceptują $x$ (dopełnienie klasy $RP$).
\end{definicja}

\begin{definicja}
    \textbf{Klasa ZPP} (zero probability of error) to $RP \cap coRP$.
\end{definicja}

\begin{definicja}
    \textbf{Klasa PP} (probabilistic polynomial time) to klasa problemów, dla których istnieje wielomianowa
    dokładna (niedeterministyczna) maszyna Turinga, która akceptuje dane wtedy i tylko wtedy,
    gdy ponad połowa obliczeń odpowiada tak (większość akceptuje dane, remin traktujemy jako odrzucenie).
\end{definicja}

\begin{twierdzenie}
    $NP \subseteq PP$
\end{twierdzenie}

\begin{lemat}
    Klasa $PP$ jest zamknięta na dopełnienie.
\end{lemat}

\begin{definicja}
    \textbf{Klasa BPP} (bounded error of probability) to klasa problemów, dla których istnieje wielomianowa dokładna
    (niedeterministyczna) maszyna Turinga, która dla każdego $x$ akceptuje $x$, jeśli co najmniej $\frac{3}{4}$ obliczeń
    odpowiada "tak", a odrzuca $x$, jeśli co najmniej $\frac{3}{4}$ obliczeń odpowiada "nie".
\end{definicja}

\begin{lemat}
    $RP \subseteq BPP \subseteq PP$
\end{lemat}

\begin{lemat}
    $BPP$ jest zamknięta na dopełnienie.
\end{lemat}
    \begin{definicja}
    Problem \textbf{QSAT} to problem prawdziwości formuł logicznych z kwantyfikatorami $Q_1,\ldots,Q_n$, bez zmiennych wolnych (zdań logicznych). $\exists_{x_1}\forall_{x_2}\exists_{x_3}\ldots Q_{nx_n}\phi(x_1,\ldots,x_n)$
\end{definicja}

\begin{twierdzenie}
    \textbf{QSAT} jest \textbf{PSPACE}-zupełny.
\end{twierdzenie}

\begin{definicja}
    \textbf{Alternująca maszyna Turinga} to niedeterministyczna maszyna $M=(Q_{\exists}\cup Q_{\forall},\Sigma,\delta,q_0)$. 
    Aby zaakceptować obliczenie dla stanów z $Q_{\exists}$ wymagamy istnienia od nich ścieżki akceptującej.
    Dla stanów z $Q_{\forall}$ - wszystkie ścieżki akceptujące.
\end{definicja}

\begin{definicja}
    \textbf{ATIME(f)} to zbiór wszystkich dokładnych, alternujących maszyn Turinga pracujących w czasie co najwyżej $f$. \\
    \textbf{ASPACE(f)} to zbiór wszystkich dokładnych, alternujących maszyn Turinga pracujących z pamięcią co najwyżej $f$.
\end{definicja}

\begin{definicja}
    $AL=ASPACE(logn),AP=\Cup_{j\in \mathbb{N}}NTIME(n^j)$
\end{definicja}

\begin{twierdzenie}
    $AP=PSPACE$
\end{twierdzenie}

\begin{twierdzenie}
    $QSAT$ jest $AP$-zupełny.
\end{twierdzenie}

\begin{definicja}
    \textbf{Monotone Circuit Value} to problem obliczenia wartościowania sieci boolowskiej złożonej tylko z bramek \textit{AND} i \textit{OR}.
\end{definicja}

\begin{twierdzenie}
    \textbf{Monotone Circuit Value} jest $P$-zupełny. \\
    \textbf{Monotone Circuit Value} jest $AL$-zupełny. \\
    $AL=P$ \\
    $ASPACE(f(n))=TIME(k^{f(n)})$
\end{twierdzenie}

\begin{definicja}
    Modyfikacja ATM - aby odczytać zawartość $i$-tej komórki, wystarczy napisać jej adres na specjalnej taśmie roboczej i wejść w odpowiedni stan.
    Dzięki temu możemy rozpatrywać obliczenia w czasie podliniowym.
\end{definicja}



    \begin{definicja}
    Rozważamy boolowskie funkcje logiczne $\{0,1\}^n\rightarrow\{0,1\}$, wyrażone w postaci sieci logicznych, jako kodowanie problemów decyzyjnych.
    \textbf{Rozmiar} sieci logicznej to liczba bramek logicznych. 
    \textbf{Głębokość} sieci to długość najdłuższej ścieżki od któregoś wejścia do wyjścia.
\end{definicja}

\begin{definicja}
    \textbf{Rodzina} sieci to nieskończony ciąg $C=\{C_1,C_2\ldots\}$ sieci logicznych, gdzie $C_n$ ma $n$ zmiennych wejściowych.
\end{definicja}

\begin{definicja}
    Język $L\subseteq\{0,1\}^*$ ma \textbf{sieci wielomianowe}, jeżeli istnieje rodzina sieci $C=\{C_1,C_2,\ldots\}$ taka, że
    1) rozmiar $C_n$ jest równy co najwyżej wartości wielomianu $p(n)$;
    2) $\forall_{x\in\{0,1\}^*}{x\in L \iff C_{|x|}(x)=1}$.
\end{definicja}

\begin{lemat}
    Każdy język z klasy $P$ ma sieć wielomianową.
\end{lemat}

\begin{lemat}
    Istnieją języki nierozstrzygalne posiadające sieci wielomianowe.
\end{lemat}

\begin{definicja}
    Rodzinę sieci $C=\{C_1,C_2\ldots\}$ nazywamy \textbf{jednostajną}, jeżeli istnieje maszyna Turinga $N\in L$ taka, że 
    dla wejścia $1^n$ generuje sieć $C_n$ (sieć musi mieć wielomianowy rozmiar w stosunku do $n$).
\end{definicja}

\begin{twierdzenie}
    Język $L$ ma jednostajną rodzinę sieci $\iff L\in P$.
\end{twierdzenie}

\begin{definicja}
    Niech $C$ jednostajna rodzina sieci i $f(n), g(n):\mathbb{N}\rightarrow\mathbb{N}$.
    Mówimy, że \textbf{równoległy czas} rodziny $C$ wynosi $f(n)$, jeżeli dla każdego $n$ głębokość sieci $C_n$ nie przekracza $f(n)$.
    Mówimy, że \textbf{całkowita praca} rodziny sieci $C$ wynosi $g(n)$, jeżeli dla każdego $n$ rozmiar sieci $C_n$ nie przekracza $g(n)$.
\end{definicja}

\begin{definicja}
    $PT/WK(f(n), g(n))$ to klasa wszystkich języków $L\in\{0,1\}^*$ dla których istnieje jednostajna rodzina sieci $C$ 
    rozstrzygająca $L$ w czasie $O(f(n))$ i wykonująca pracę $O(g(n))$.
\end{definicja}

\begin{lemat}
    Jeżeli $L\in PT/WK(f(n),g(n))$, to istnieje maszyna Turinga $M\in L$ generująca program na maszynę $PRAM$
    obliczającą $L$ w czasie $O(f(n))$ przy użyciu $O(g(n)/f(n))$ procesorów.
\end{lemat}

\begin{definicja}
    Klasa \textbf{NC} - równoległe obliczenia w czasie polilogarytmicznym (z pracą wielomianową).
    $NC_i = PT/WK(log^in, n^k)$
    $NC = \bigcup_{i\in\mathbb{N}}{NC_i}$
\end{definicja}


    \begin{definicja}
    \textbf{Aukcją} nazywamy grę (domyślnie w postaci strategicznej), gdzie każdy z graczy $p$ składa (w tajemnicy)
    pewną ofertę $w_p$ za jakieś dobro, które według nich jest warte $v_p$. Na podstawie strategii graczy (ofert)
    i wartości licytowanego przedmiotu, podawane są wypłaty $u$ oraz zwycięzcy aukcji (którzy dostają dobro).
    Z reguły jest jeden zwycięzca, ale niekoniecznie musi mieć on największą wypłatę. Z puntku widzenia teorii gier, zwycięzcy nie są istotni.
\end{definicja}

\begin{definicja}{\textbf{Aukcja pierwszej cen}y}
    Zwycięzcą zostaje gracz, który wylicytował najwyżej, i tyle też płaci. Wobec tego
    \[u_p = \begin{cases} 
                v_p - w_p & \text{jeśli } w_p = \max\limits_{i \in P}w_i \\
                0 & \text{w p. p.}
            \end{cases}
    \]
\end{definicja}

\begin{definicja}{\textbf{Aukcja Vickreya} (aukcja drugiej kwoty)}
    Dla każdego gracza $i$ przedmiot wystawiany na aukcji ma wartość $v_i$.
    $u_i(s) = v_i - p$, jeśli strategia $s$ prowadzi do wygranej gracza $i$, a $u_i(s) = p$, gdy $s$ oznacza,
    że aukcję wygrał inny gracz. Strategia gracza to stawiana kwota $p$. Każdy gracz podaje swoją kwotę w kopercie
    w tajemnicy (one-shot game). Wygrywa gracz z najwyższą ofertą w kopercie, ale płaci kwotą drugiego rekordu (drugiej najwyższej kwoty).
\end{definicja}

\begin{twierdzenie}{(Vickrey)}
    Zagranie $w_p = v_p$ w Aukcji Vickreya jest optymalne.
\end{twierdzenie}

\begin{definicja}
    Niech $V_i$ oznacza zbiór możliwych wartości licytowanych przedmiotów przez gracza $i$.
    $V = \bigtimes\limits_{i \in P} V_i$ oraz $V_{-i} = \bigtimes\limits_{j \in P \setminus \{i\}} V_j$.
    \textbf{Mechanizm bezpośredniego ujawnienia} (ang. direct revelation mechanism) $(f, p)$ składa się z funkcji wyboru społecznego
    (ang. social choice function) $f : V \rightarrow A$, gdzie $A$ to zbiór możliwych rozstrzygnięć, dla których gracze
    wyznaczają waluacje $v_i$, oraz $p_i : V \rightarrow \mathbb{R}$, które są płatnościami graczy za grę w danej aukcji.
    Funkcją dobra społecznego (ang. social welfare function) nazywamy funkcję $F \in \mathbb{R}^A$, określoną wzorem $F(a) = \sum\limits_{i \in P} v_i(a)$.
\end{definicja}

\begin{definicja}
    Mówimy, że mechanizm $(f, p_1, ..., p_n)$ jest \textbf{zgodny z motywacją} (ang. incentive compatible), jeśli
    $(\forall i \in P)(\forall V_i)(\forall V_{-i})f(\langle v_i, v_{-i} \rangle) - p_i(\langle v_i, v_{-i} \rangle) \geq f(\langle v_i', v_{-i} \rangle) - p_i(\langle v_i', v_{-i} \rangle)$
\end{definicja}

\begin{definicja}
    \textbf{Mechanizm Vickreya-Clarke'a-Groovesa} to mechanizm bezpośredniego ujawnienia, jeśli $f(v) \in arg\max\limits_{a \in A}(\sum\limits_{i \in P}v_i(a))$
    i istnieją funkcje (dla $i \in P$) $h_i : V_{-i} \rightarrow \mathbb{R}$, że $(\forall v \in V) p_i(v) = h_i(v_{-i}) - \sum\limits_{j \neq i}v_j(f(v))$.
\end{definicja}

\begin{twierdzenie}{(VCG)}
    Każdy mechanizm VCG jest zgodny z motywacją.
\end{twierdzenie}

\begin{definicja}
    Mówimy, że mechanizm jest \textbf{indywidualnie racjonalny}, jeśli $(\forall i \in P) v_i(f(v)) - p_i(v) \geq 0$.
    Mówimy, że mechanizm \textbf{nie ma pozytywnego przepływu}, jeśli $(\forall i \in P)p_i(v) \geq 0$.
\end{definicja}

\begin{definicja}{\textbf{Reguła Clarke'a} (Clarke's pivot rule)}
    $h_i(v_{-i}) = \max\limits_{b \in A}\sum\limits_{j \neq i}v_i(b)$ nazywamy wypłatą Clarke'a.
    Wtedy $p_i(v) = \max\limits_{b \in A}(v_i(b) - v_i(a))$, gdzie $a = f(v)$.
\end{definicja}

\begin{twierdzenie}{(Clarke)}
    Mechanizm VCG z regułą Clarke'a nie ma pozytywnych przepływów. Jeśli $(\forall v_i \in V_i)(\forall a \in A)v_i(a) \geq 0$, to jest indywidualnie racjonalny.
\end{twierdzenie}
    \begin{twierdzenie}{(Algorytm alfa-beta)}
    2-os. EFG o sumie stałej. $minmax(v,lev)=$ return $alfabeta(v,lev,-\infty,\infty)$. Zwraca wartość gry dla równowagi doskonałej w grze o pełnej informacji.
\end{twierdzenie}
\begin{algorithm}[H]
    \SetAlgoLined
    \LinesNumberedHidden
    \KwData{Input data if necessary}
    
    \If{$v\in L$}{
        return $u(v)$
    }
    \If{$P_V(v)=1$}{
        \For{$s\in succ(v)$}{
            $\alpha=\max(\alpha,Alfabeta(w,lev+1,\alpha,\beta))$\\
            \If{$\alpha\geq\beta$}{
                odcinamy gałąź Beta
            }
        }
        return $\alpha$
    }
    \If{$P_V(v)=2$}{
        \For{$s\in succ(v)$}{
            $\beta=\min(\beta,Alfabeta(w,lev+1,\alpha,\beta))$\\
            \If{$\alpha\geq\beta$}{
                odcinamy gałąź Alpha
            }
        }
        return $\beta$
    }
    \caption{Alfabeta(v,lev,$\alpha,\beta$)}
\end{algorithm}
\begin{definicja}{\textbf{Gra koalicyjna/kooperacyjna, z wypłatami ubocznymi}}
    $(P,v)$, $P$ sk. zb. graczy, $v:\mathcal{P}(P)\rightarrow\mathbb{R}$, $v(\emptyset)=0$, $v(S)$ zadaje 
    \textbf{siłę koalicyjną (wartość koalicyjną)} koalicji $S\subseteq P$, gdzie $v(S)$ ma być rozdystrybuowana między graczy w koalicji $S$.
    $P$ - \textbf{wielka koalicja}.
\end{definicja}
\begin{definicja}{\textbf{Gra kooperacyjna jest}}
    \textbf{nad-addytywna: } $(\forall S,T\subseteq P) ((S\cap T=\emptyset)\rightarrow(v(S)+v(T)\leq v(S\cup T)))$. Oznacza opłacalność łąćzenia się w większe koalicje.
    \textbf{monotoniczna: } $(\forall S\subseteq T\subseteq P) (v(S)\leq v(T))$.
    \textbf{wypukła: } $(\forall S,T\subseteq P) (v(S)+v(T)-v(S\cap T)\leq v(S\cup T))$. Gracze mają niemniejsze wypływy w większych koalicjach. Wypukła$\rightarrow$nad-addytywna.
\end{definicja}
\begin{fakt}\label{w10-f-normkoal}
    $(P=[n],S,u)$ gra w postaci normalnej. Wtedy $(P,v)$ grą koalicyjną, gdzie dla $T\subset P$: 
    $v(T)=\max\limits_{\pi_T\in\Delta_T^*}\min\limits_{\pi_{-T}\in\Delta_{-T}^*}\sum\limits_{i\in T}{u_i(\langle \pi_T,\pi_{-T} \rangle)}$.
\end{fakt}
\begin{fakt}
    Każdą grę nad-addytywną można otrzymać z pewnej gry w postaci normalnej za pomocą wzoru w Fakcie~\ref{w10-f-normkoal}.
\end{fakt}
\begin{definicja}
    $P=[n]$. Wektor wypłat w grze koalicyjnej to $x\in\mathbb{R}^n$, jest on:
    \textbf{racjonalny grupowo (alokacją/podziałem)} jeśli $\sum_{i=1}^n{x_i=v(N)}$;
    \textbf{racjonalny indywidualnie} jeśli $(\forall i\in [n]) (v(\{i\}) \leq x_i)$;
    \textbf{racjonalny koalicyjnie (stabilny)} jeśli $(\forall S\subseteq P)(v(S)\leq\sum_{i\in S}{x_i})$;
    \textbf{imputacją} jeśli jest indywidualnie racjonalną alokacją.
\end{definicja}
\begin{fakt}
    Nad-addytywna gra koalicyjna posada podział.
\end{fakt}
\begin{definicja}
    Zbiór $C(v)$ stabilnych podziałów to \textbf{rdzeń gry}. Może być pusty. Jest domknięty i wypukły.
\end{definicja}



    % \DeclareSymbolFont{symbolsC}{U}{txsyc}{m}{n}
% \DeclareMathSymbol{\notniFromTxfonts}{\mathrel}{symbolsC}{61}

\begin{definicja}
    Gracze $i,j\in P$ są \textbf{zamienni względem $v$} jeśli $(\forall S\subseteq P)((i,j\notin S)\rightarrow(v(S\cup\{i\})=v(S\cup\{j\})))$.
    Gracz $i$ jest \textbf{nieistotny względem $v$} gdy $(\forall S\not\owns i)(v(S)=v(S\cup\{i\}))$.
    Gracz $i$ ma \textbf{prawo veta} jeśli $v(P)\textbackslash \{i\}=0 \land v[\mathcal{P}(P)]\subset[0,\infty]$.
    Jeśli $v[\mathcal{P}(P)\subset\{0,1\}]$, to $(N,v)$ jest \textbf{grą prostą}; 
    równoważnie: gra koalicyjna jest prosta gdy $(\forall S\subseteq P)(v(S)\in\{0,1\})$.
\end{definicja}
\begin{fakt}
    Koalicyjna gra prosta ma niepusty rdzeń IFF ma graczy z prawem veta. Składa się on wtedy z podziałów, gdzie gracze z prawem veta odstają sumaryczną wypłatę 1.
\end{fakt}
\begin{definicja}{\textbf{Rozwiązanie gry}}
    Dla gry $(P,v)$ pewien zbiór podziałów $\Phi(v)$ zależny od wyboru $v$. 
    Jeśli rozwiązaie jest 1-el., to $\Phi(v)$ nazywamy \textbf{wartością gry}, a $\Phi_i(v)$ wartością $i$-tego gracza.
    % Czasami $\Phi$ jest postrzegane jako multifunkcja 
\end{definicja}
\begin{definicja}{\textbf{Wartość Shapleya (SV)}}
    Wartość gry dla $i\in P$: $\Phi_i(v)=\sum\limits_{S\subseteq P\textbackslash\{i\}}{\frac{|S|!(|P\textbackslash S|-1)!}{|P|!}(v(S\cup\{i\})-v(S))}$
    Mówi jaki jest średni wkład w siłę koalicyjną gracza $i$ po wszystkich możliwych liniowych porządkach dodawania pojedynczych graczy do koalicji.
\end{definicja}
\begin{fakt}
    W grach wypukłych, wartość Shapleya jest w rdzeniu gry.
\end{fakt}
\begin{fakt}{\textbf{Własności SV}}
    \textbf{Wydajność:} SV jest podziałem.
    \textbf{Symetria:} Dla wszystkich $i,j$ zamiennych względem $v$ mamy $\Phi_i(v)=\Phi_j(v)$.
    \textbf{Liniowość:} $\Phi_i(av+bw)=a\Phi_i(v)+b\Phi_i(w)$.
    \textbf{Nieistotny gracz:} Jeśli $i$ nieistotny względem $v$, to $\Phi_i(v)=0$.
    \textbf{Addytywność:} Gdy $v$ jest nad-addytywna, to $\Phi(v)$ indywiidualnie racjonalna; 
    jeśli $v$ pod-addytywna, to $(\forall i) \Phi_i(v)\leq v(\{i\})$;
    jeśli addytywna, zachodzi równość. \\
    Zbiór pewnych cech, których używamy do predykcji możemy uznać za graczy. Siły koalicji będą
wskazywały na jakość predykcji. Wtedy wartość Shapleya można użyć w kontekście analizy wariancji ANOVA, czy w machine
learningu (np. LIME) Podobny model można wykorzystać w szeregach czasowych do zdefiniowania slidShap (X 2023), gdzie
siły koalicyjne są wyznaczane przy pomocy entropii Shannona, a który to służy do predykcji przyszłych wartości ciągu
wektorów losowych.
\end{fakt}
\begin{twierdzenie}{(Bondareva-Shapley)}
    Dla gry koalicyjnej, posiadanie niepustego rdzeń jest równoważne z warunkiem: 
    $(\forall i\in P) (\forall \alpha:\mathcal{P}(P)\rightarrow[0,1]) \left(\sum\limits_{S\owns i}{\alpha(S)=1} \right)\rightarrow \left(\sum\limits_S{\alpha(S)v(S)\leq v(P)} \right)$.
\end{twierdzenie}
\begin{definicja}{(\textbf{Indeksy siły Shapleya-Shubika i Banzhafa})}
    Rozważmy prostą grę. $v(S)=1$ oznacza, że można uformować koalicję $S$, która będzie rządzić (wygrywać wybory). 
    Zakładamy że $v(P)=1$, $v(\emptyset)=0$, $v$ niemalejąca, dla $S\subseteq P$ jesli $v(S)=1$ to $v(P\textbackslash S)=0$.
    \textbf{Indeksem siły Shapleya-Shubika} $i$-tego gracza $Sh(i)$ jest wartość Shapleya $i$-tego gracza w takiej grze.
    Niech $S\subseteq P$. Określamy $K(S)=\{i\in S: v(S)=1, v(S\textbackslash\{i\})=0\}$ - zbiór kluczowych koalicjantów w koalicji $S$.
    \textbf{Indeksem Banzhafa} gracza $i$ nazywamy: $B(i)=\frac{|\{S\subseteq P: i\in K(S)\}|}{\sum\limits_{j\in P}|\{S\subseteq P: j\in K(S)\}|}$.
    W grach prostych $\sum\limits_{i\in P}{Sh(i)} = \sum\limits_{i\in P}{B(i)} = 1$.
\end{definicja}
% TODO: DODAĆ UWAGĘ 79 (PARADOKS DE CONDORCETA)



    \input{wyklady/wyklad12}
    \input{wyklady/wyklad13}
    \input{wyklady/wyklad14}
    \input{wyklady/wyklad15}
\end{multicols*}

\end{document}