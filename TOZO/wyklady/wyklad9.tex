\begin{definicja}
    Niech $Q$ będzie wielomianowo zrównoważoną, wielomianowo rozstrzygalną relacją binarną. 
    \textbf{Problem zliczania} związany z relacją $Q$ to następujące pytanie:
    Ile istnieje dla zadanego $x$ różnych $y$ takich, że $(x,y)\in Q$?
    (Wynik podajemy w postaci liczby binarnej.)
\end{definicja}

\begin{definicja}
    Klasa \textbf{\#P} zawiera wszystkie problemy zliczania związane z wielomianowo zrównoważonymi, wielomianowo rozstrzygalnymi relacjami.
\end{definicja}
% TODO: podać przykład z \#Matching

\begin{twierdzenie}
    $NP\subseteq \#P$
\end{twierdzenie}

\begin{definicja}
    Problem zliczający $A$ \textbf{redukuje się} do problemu $B$, jeśli istnieją funkcje $R,S\in L$ takie, że
    1) $x\in A \iff R(x)\in B$,
    2) $m$ jest odpowiedzią dla $R(x)$ w $B$ $\iff$ $S(m)$ jest odpowiedzią dla $x$ w $A$.
\end{definicja}

\begin{twierdzenie}
    $\#SAT$ jest $\#P$-zupełny. (wystarczy zmodyfikować tw. Cook'a.)
\end{twierdzenie}

\begin{twierdzenie}
    $\#HamiltonPath$ jest $\#P$-zupełny. (bo redukcja $3SAT$ do $HP$ była oszczędna.)
\end{twierdzenie}

\begin{twierdzenie}
    $Permanent$ jest $\#P$-zupełny.
\end{twierdzenie}

\begin{twierdzenie}
    $\#2SAT$ jest $\#P$-zupełny.
\end{twierdzenie}

\begin{definicja}
    Klasa $\oplus P$ zawiera wszystkie problemy zliczania w których pytamy o nieparzystość liczby rozwiązań.
\end{definicja}

\begin{twierdzenie}
    $\oplus SAT$ jest $\oplus P$-zupełny.
\end{twierdzenie}

\begin{lemat}
    $\oplus P$ jest zamknięta na dopełnienie, tj. $\oplus P = co-\oplus P$.
\end{lemat}


