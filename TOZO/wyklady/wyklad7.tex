\begin{definicja}
    Problem \textbf{QSAT} to problem prawdziwości formuł logicznych z kwantyfikatorami $Q_1,\ldots,Q_n$, bez zmiennych wolnych (zdań logicznych). $\exists_{x_1}\forall_{x_2}\exists_{x_3}\ldots Q_{nx_n}\phi(x_1,\ldots,x_n)$
\end{definicja}

\begin{twierdzenie}
    \textbf{QSAT} jest \textbf{PSPACE}-zupełny.
\end{twierdzenie}

\begin{definicja}
    \textbf{Alternująca maszyna Turinga} to niedeterministyczna maszyna $M=(Q_{\exists}\cup Q_{\forall},\Sigma,\delta,q_0)$. 
    Aby zaakceptować obliczenie dla stanów z $Q_{\exists}$ wymagamy istnienia od nich ścieżki akceptującej.
    Dla stanów z $Q_{\forall}$ - wszystkie ścieżki akceptujące.
\end{definicja}

\begin{definicja}
    \textbf{ATIME(f)} to zbiór wszystkich dokładnych, alternujących maszyn Turinga pracujących w czasie co najwyżej $f$. \\
    \textbf{ASPACE(f)} to zbiór wszystkich dokładnych, alternujących maszyn Turinga pracujących z pamięcią co najwyżej $f$.
\end{definicja}

\begin{definicja}
    $AL=ASPACE(logn),AP=\bigcup_{j\in \mathbb{N}}NTIME(n^j)$
\end{definicja}

\begin{twierdzenie}
    $AP=PSPACE$
\end{twierdzenie}

\begin{twierdzenie}
    $QSAT$ jest $AP$-zupełny.
\end{twierdzenie}

\begin{definicja}
    \textbf{Monotone Circuit Value} to problem obliczenia wartościowania sieci boolowskiej złożonej tylko z bramek \textit{AND} i \textit{OR}.
\end{definicja}

\begin{twierdzenie}
    \textbf{Monotone Circuit Value} jest $P$-zupełny. \\
    \textbf{Monotone Circuit Value} jest $AL$-zupełny. \\
    $AL=P$ \\
    $ASPACE(f(n))=TIME(k^{f(n)})$
\end{twierdzenie}

\begin{definicja}
    Modyfikacja ATM - aby odczytać zawartość $i$-tej komórki, wystarczy napisać jej adres na specjalnej taśmie roboczej i wejść w odpowiedni stan.
    Dzięki temu możemy rozpatrywać obliczenia w czasie podliniowym.
\end{definicja}


