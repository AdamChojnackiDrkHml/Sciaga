\begin{definicja}
    Deterministyczna maszyna Turinga $M$ to dwustronnie nieskończona taśma oraz czwórka $(Q, \Sigma, \delta, q_0)$, gdzie:
    \begin{itemize*}[label={}]
        \item $Q$ --- skończony zbiór stanów (bez \textbf{tak, nie h}),
        \item $\Sigma$ --- alfabet maszyny (poza $\sqcup$ domyślnym znakiem na maszynie),
        \item $\delta$ --- funkcja przejścia (wiadomo jak wygląda)
        \item $q_0 \in Q$ --- stan początkowy
    \end{itemize*}
\end{definicja}
\begin{definicja}
    Dla \textit{TM} $M$ i ciągu początkowego $x \in \Sigma^{*}$ przez $M(x)$
    oznaczamy wynik działania $M$ na $x$. W szczególności:
    \begin{itemize*}[label={}]
        \item $M(x) = tak$ --- maszyna kończy w stanie akceptującym,
        \item $M(x) = nie$ --- maszyna kończy w stanie odrzucającym,
        \item $M(x) = \nearrow$ --- maszyna nie zatrzymała się,
        \item $M(x) = y$ --- maszyna kończy pracę w \textbf{h} i $y \in (\Sigma \cup \{\sqcup\})^{*}$ jest zawartością taśmy bez blanków z lewej i prawej.
    \end{itemize*}
\end{definicja}
\begin{definicja}
    Konfiguracją maszyny $M$ będziemy nazywać trójkę $(q, \alpha, \beta)$,
    gdzie $\alpha, \beta \in {(\Sigma \cup \{\sqcup\})}^{*}$, 
    $\alpha$ symbole na taśme widziane przez maszynę bez zbędnych blanków po lewej,
    $\beta$ symbole na taśme widziane przez maszynę bez zbędnych blanków po prawej,
    $q$ aktualny stan maszyny.
    $\stackrel{M}{\rightarrow}$ definiujemy jako relację przejścia w jednym kroku między dwoma konfiguracjami maszyny $M$.
    $\stackrel{M^i}{\rightarrow}$ i $\stackrel{M^{*}}{\rightarrow}$ to odpowiednio przejście w $i$ krokach i przejście w dowolnej liczbie kroków.
\end{definicja}
\begin{definicja}
    $k$-taśmowa \textit{TM} ma funkcję przejścia w postaci:
    $\delta : \Cal{Q} \times {(\Sigma \cup \{\sqcup\})}^{k} \rightarrow$ $(\Cal{Q} \cup \{tak, nie, h\}) \times$ ${[(\Sigma \cup \{\sqcup\}) \times \{\leftarrow, \rightarrow, -\}]}^k$
\end{definicja}
\begin{twierdzenie}
    Dla dowolnej \textit{TM} $M$ z $k$ taśmami można skonstruować jednotaśmową \textit{TM} $M'$ taką, że:
    \begin{enumerate*}[label=\roman*)]
        \item dla każdego słowa wejściowego $x$ zachodzi $M(x) = M'(x)$ (maszyny są równoważne),
        \item jeśli $M$ na słowie wejściowym $x$ wykonała $f(|x|)$ kroków, to $M'$ wykona co najwyżej $O({(f(|x|))}^2)$.
    \end{enumerate*}
    Zwiększenie liczby taśmy, nie zwiększa możliwości obliczeniowych maszyn Turinga.
\end{twierdzenie}
\begin{twierdzenie}
    Dla dowolnej \textit{TM} $M$ pracującej w czasie $f(|x|)$ i dowolnego $\varepsilon > 0$
    można skonstruować równoważną \textit{TM} $M'$ pracującą w czasie $f'(|x|) = \varepsilon f(|x|) + |x| + 2$.
\end{twierdzenie}
\begin{definicja}
    Niedeterministyczna \textit{TM} ma funkcję przejścia w postaci:
    $\delta : \Cal{Q} \times {(\Sigma \cup \{\sqcup\})} \rightarrow$ $2^{(\Cal{Q} \cup \{tak, nie, h\}) \times {[(\Sigma \cup \{\sqcup\}) \times \{\leftarrow, \rightarrow, -\}]}}$
\end{definicja}
\begin{twierdzenie}
    Dla dowolnej jednostaśmowej \textit{NTM} $M$ można skonstruować dwutaśmową \textit{DTM} $M'$ taką, że:
    \begin{enumerate*}[label=\roman*)]
        \item Dla każdego słowa wejściowego $x$ zachodzi $M(x) = M'(x)$ (maszyny są równoważne).
        \item Jeśli $M$ na słowie wejściowym $x$ wykonała $f(|x|)$ kroków, to $M'$ wykona co najwyżej $O(c^{f(|x|)})$ kroków dla pewnej stałej $c$.
    \end{enumerate*}
    Niedeterminizm nie zwiększa możliwości obliczeniowych maszyn Turinga.
\end{twierdzenie}
\begin{definicja}
    Niech $L \subseteq \Sigma^*$. Niech \textit{TM} $M$, taka że dla każdego $x \in \Sigma^{*}$,
    jeżeli $x \in L$ to $M(x) = tak$,
    jeżeli $x \notin L$ to $M(x) = nie$.
    Wówczas mówimy, że $M$ rozstrzyga $L$ ($L$ nazywamy rozstrzygalnym/rekurencyjnym).
\end{definicja}
\begin{definicja}
    Niech $L \subseteq \Sigma^*$. Niech \textit{TM} $M$, taka że dla każdego $x \in \Sigma^{*}$,
    jeżeli $x \in L$ to $M(x) = tak$,
    Wówczas mówimy, że $M$ rozpoznaje $L$ ($L$ nazywamy rozpoznawalnym/rekurencyjnie przeliczalnym).
    Język rozpoznawany przez $M$ oznaczamy jako $L(M)$.
\end{definicja}
\begin{definicja}
    Niech $f : \Sigma^{*} \rightarrow {(\Sigma \cup \{\sqcup\})}^{*}$.
    Mówimy, że \textit{TM} $M$ oblicza $f$, jeżeli dla każdego $x \in \Sigma^{*}$ mamy $M(x) = f(x)$.
    $f$ nazywamy wtedy funkcją rekurencyjną.
\end{definicja}
\begin{lemat}
    Suma i przekrój języków rozpoznawalnych (rozstrzygalnych) jest językiem rozpoznawalnym (rozstrzygalnym).
\end{lemat}
\begin{lemat}
    Jeśli $L$ jest językiem rozstrzygalnym to $L'$ też.
\end{lemat}
\begin{lemat}
    $L$ jest rozstrzygalny $\Leftrightarrow$ $L$ i $L'$ są rozpoznawalne.
\end{lemat}
\begin{definicja}
    \textit{TM} $M$ nazywamy generatorem języka $L$, jeśli startując z pustym wejściem na specjalnej taśmie wyjściowej,
    wypisze wszystkie słowa z języka $L$ (każde słowo pojawi się na tej taśmie kiedyś).
    Język generowany przez $M$ oznaczamy jako $G(M)$.
\end{definicja}
\begin{lemat}
    Jeśli dla języka $L$ istnieje generator $G$, to $L$ jest rozpoznawalny.
\end{lemat}
\begin{lemat}
    Jeśli $M$ rozpoznaje $L$, to istnieje generator $G$ dla $L$.
\end{lemat}
\begin{lemat}
    Jeśli $L$ jest rozstrzygalny to istnieje generator wypisujący słowa z $L$
    w porządku leksykograficznym.
\end{lemat}
\begin{definicja}
    Właściwą funkcją złożoności nazywamy taką niemalejącą funkcję $f: \bb{N} \rightarrow \bb{N}$
    dla której istnieje \textit{DTM} która:
    \begin{enumerate*}[label=\roman*)]
        \item Startując z napisem $0^n$ na wejściu(read-only) kończy z napisem $0^{f(n)}$ na wyjściu(write-only).
        \item Pracuje w czasie $\leq O(n + f(n))$.
        \item Używa na taśmach roboczych $\leq O(f(n))$ komórek.
    \end{enumerate*}
\end{definicja}
