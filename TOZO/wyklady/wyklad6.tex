\begin{definicja}
    Wielomianowa dokładna (niedeterministyczna) maszyna Turinga jest typu
    \textbf{Monte Carlo} dla problemu $A$, jeśli dla każdego $x \in A$ co najmniej
    połowa obliczeń na słowie $x$ jest akceptująca, a dla $x \notin A$
    wszystkie obliczenia odrzucają $x$.
\end{definicja}

\begin{definicja}
    \textbf{Klasa RP} (randomized polynomial time) to zbiór problemów posiadających
    maszyny Turinga typu Monte Carlo.
\end{definicja}

\begin{lemat}
    Jeżeli dokładna wielomianowa maszyna Turina $M$ dla każdego pozytywnego przykładu $x$
    ma co najmniej $0 < \varepsilon < \frac{1}{2}$ obliczeń akceptujących, to $M$ jest typu Monte Carlo.
\end{lemat}

\begin{lemat}
    $P \subseteq RP \subseteq NP$
\end{lemat}

\begin{definicja}
    \textbf{Klasa coRP} to zbiór problemów posiadających wielomianową dokładną (niedeterministyczną)
    maszynę Turinga taką, że dla każdego negatywnego przykładu $x$ co najmniej połowa obliczeń odrzuci $x$,
    a dla pozytywnego $x$ wszystkie obliczenia zaakceptują $x$ (dopełnienie klasy $RP$).
\end{definicja}

\begin{definicja}
    \textbf{Klasa ZPP} (zero probability of error) to $RP \cap coRP$.
\end{definicja}

\begin{definicja}
    \textbf{Klasa PP} (probabilistic polynomial time) to klasa problemów, dla których istnieje wielomianowa
    dokładna (niedeterministyczna) maszyna Turinga, która akceptuje dane wtedy i tylko wtedy,
    gdy ponad połowa obliczeń odpowiada tak (większość akceptuje dane, remin traktujemy jako odrzucenie).
\end{definicja}

\begin{twierdzenie}
    $NP \subseteq PP$
\end{twierdzenie}

\begin{lemat}
    Klasa $PP$ jest zamknięta na dopełnienie.
\end{lemat}

\begin{definicja}
    \textbf{Klasa BPP} (bounded error of probability) to klasa problemów, dla których istnieje wielomianowa dokładna
    (niedeterministyczna) maszyna Turinga, która dla każdego $x$ akceptuje $x$, jeśli co najmniej $\frac{3}{4}$ obliczeń
    odpowiada "tak", a odrzuca $x$, jeśli co najmniej $\frac{3}{4}$ obliczeń odpowiada "nie".
\end{definicja}

\begin{lemat}
    $RP \subseteq BPP \subseteq PP$
\end{lemat}

\begin{lemat}
    $BPP$ jest zamknięta na dopełnienie.
\end{lemat}