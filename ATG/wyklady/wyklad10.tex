\begin{twierdzenie}{(Algorytm alfa-beta)}
    2-os. EFG o sumie stałej. $minmax(v,lev)=$ return $alfabeta(v,lev,-\infty,\infty)$. Zwraca wartość gry dla równowagi doskonałej w grze o pełnej informacji.
\end{twierdzenie}
\begin{algorithm}[H]
    \SetAlgoLined
    \LinesNumberedHidden
    \KwData{Input data if necessary}
    
    \If{$v\in L$}{
        return $u(v)$
    }
    \If{$P_V(v)=1$}{
        \For{$s\in succ(v)$}{
            $\alpha=\max(\alpha,Alfabeta(w,lev+1,\alpha,\beta))$\\
            \If{$\alpha\geq\beta$}{
                odcinamy gałąź Beta
            }
        }
        return $\alpha$
    }
    \If{$P_V(v)=2$}{
        \For{$s\in succ(v)$}{
            $\beta=\min(\beta,Alfabeta(w,lev+1,\alpha,\beta))$\\
            \If{$\alpha\geq\beta$}{
                odcinamy gałąź Alpha
            }
        }
        return $\beta$
    }
    \caption{Alfabeta(v,lev,$\alpha,\beta$)}
\end{algorithm}
\begin{definicja}{\textbf{Gra koalicyjna/kooperacyjna, z wypłatami ubocznymi}}
    $(P,v)$, $P$ sk. zb. graczy, $v:\mathcal{P}(P)\rightarrow\mathbb{R}$, $v(\emptyset)=0$, $v(S)$ zadaje 
    \textbf{siłę koalicyjną (wartość koalicyjną)} koalicji $S\subseteq P$, gdzie $v(S)$ ma być rozdystrybuowana między graczy w koalicji $S$.
    $P$ - \textbf{wielka koalicja}.
\end{definicja}
\begin{definicja}{\textbf{Gra kooperacyjna jest}}
    \textbf{nad-addytywna: } $(\forall S,T\subseteq P) ((S\cap T=\emptyset)\rightarrow(v(S)+v(T)\leq v(S\cup T)))$. Oznacza opłacalność łąćzenia się w większe koalicje.
    \textbf{monotoniczna: } $(\forall S\subseteq T\subseteq P) (v(S)\leq v(T))$.
    \textbf{wypukła: } $(\forall S,T\subseteq P) (v(S)+v(T)-v(S\cap T)\leq v(S\cup T))$. Gracze mają niemniejsze wypływy w większych koalicjach. Wypukła$\rightarrow$nad-addytywna.
\end{definicja}
\begin{fakt}\label{w10-f-normkoal}
    $(P=[n],S,u)$ gra w postaci normalnej. Wtedy $(P,v)$ grą koalicyjną, gdzie dla $T\subset P$: 
    $v(T)=\max\limits_{\pi_T\in\Delta_T^*}\min\limits_{\pi_{-T}\in\Delta_{-T}^*}\sum\limits_{i\in T}{u_i(\langle \pi_T,\pi_{-T} \rangle)}$.
\end{fakt}
\begin{fakt}
    Każdą grę nad-addytywną można otrzymać z pewnej gry w postaci normalnej za pomocą wzoru w Fakcie~\ref{w10-f-normkoal}.
\end{fakt}
\begin{definicja}
    $P=[n]$. Wektor wypłat w grze koalicyjnej to $x\in\mathbb{R}^n$, jest on:
    \textbf{racjonalny grupowo (alokacją/podziałem)} jeśli $\sum_{i=1}^n{x_i=v(N)}$;
    \textbf{racjonalny indywidualnie} jeśli $(\forall i\in [n]) (v(\{i\}) \leq x_i)$;
    \textbf{racjonalny koalicyjnie (stabilny)} jeśli $(\forall S\subseteq P)(v(S)\leq\sum_{i\in S}{x_i})$;
    \textbf{imputacją} jeśli jest indywidualnie racjonalną alokacją.
\end{definicja}
\begin{fakt}
    Nad-addytywna gra koalicyjna posada podział.
\end{fakt}
\begin{definicja}
    Zbiór $C(v)$ stabilnych podziałów to \textbf{rdzeń gry}. Może być pusty. Jest domknięty i wypukły.
\end{definicja}


