\begin{definicja}
    Jeśli gra $G$ spełnia warunek: $(\forall s \in S)(\sum\limits_{i \in [n]}u_i = 0)$,
    to taką grę nazywamy \textbf{gra o sumie zerowej}. 
    Jeśli $K$ jest ciałem skalarów nad $V$, to grę $G$, która spełnia warunek 
    $(\exists c \in K)(\forall s \in S)((\sum\limits_{i \in [n]}u_i = c))$
    nazywamy \textbf{grą o sumie stałej}.
\end{definicja}
\begin{fakt}
    Każda gra o sumie zerowej jest grą o sumie stałej.
\end{fakt}
\begin{twierdzenie}
    Każda gra o sumie niezerowej na $n$ graczy, jest równoważna z grą o sumie zerowej dla $n+1$ graczy.
\end{twierdzenie}
\begin{fakt}
    W grzez o sumie stałej, każda strategia jest Pareto-optymalna.
\end{fakt}
\begin{definicja}
    Rozważmy grę dwuosobową o sumie stałej $c$, w której $S_1, S_2$ są zwartymi podzbiorami $\bb{R}^k$
    (dla pewnego $k \in \bb{N}$).
    \textbf{Wartością dolną} gry $\underline{v}$ nazywamy $\sup_{s_1 \in S_1} \inf_{s_2 \in S_n} u_1(s_1, s_2)$.
    \textbf{Wartością górną} gry $\overline{v}$ nazywamy $\inf_{s_1 \in S_1} \sup_{s_2 \in S_n} u_1(s_1, s_2)$.
    Jeśli $\overline{v} = \underline{v}$, to mówimy, że gra ma \textbf{wartość} $v$.
    Profil strategii, który realizuje wartość gry, nazywamy \textbf{punktem siodłowym}.
    Profil strategii $(s_1, s_2)$ taki, że $s_1$ realizuje wartość dolną, a $s_2$ górną, nazywamy \textbf{strategią minimaksową/maksiminową}
\end{definicja}
\begin{twierdzenie}
    Dla gry dwuosobowej o sumie stałej:
    \begin{enumerate*}[label=\roman*)]
        \item $\underline{v} \leq \overline{v}$,
        \item Gra ma równowagę Nasha $\Rightarrow$ $\underline{v} = \overline{v}$ i każdy punkt siodłowy jest równowagą Nasha.
    \end{enumerate*}
\end{twierdzenie}
\begin{definicja}
    Profile strategii, które mają wartość z przedziału $[\underline{v}, \overline{v}]$, nazywamy \textbf{strategiami bezpieczeństwa}.
\end{definicja}
\begin{definicja}
    Niech $(P = [n], S, u)$, będzie grą skończoną w postaci normalnej.
    \textbf{Strategią mieszaną gracza} $i$ nazywamy rozkład prawdopodobieństwa $\pi_i : S_i \rightarrow [0, 1]$.
    \textbf{Strategią mieszaną gry} nazywamy rozkład produktowy $\pi := \bigtimes_{i=0}^{n}\pi_{i} : S \rightarrow [0, 1]$.
    Wypłatą gracza $j$ jest wtedy wartość oczekiwana: $u_j(\pi) = \sum\limits_{(s_j \in S_j)}\sum\limits_{s_{-j} \in S_{-j}} u_j(\langle s_j, s_{-j} \rangle)\pi(\langle s_j, s_{-j} \rangle)$.
    Zbiór wszystkich rozkładów gracza $i$ będziemy oznaczać przez $\Delta_i$, a ich produkt przez $\Delta$.
\end{definicja}
\begin{definicja}
    \textbf{Mieszaną równowagą Nasha} nazywamy rozkład łączny $\pi^{*}$ taki, że:
    $(\forall i \in [n])(\forall \pi_i \in \Delta_i)(u_i(\pi^{*}) \geq u_i(\langle \pi_i, \pi_{-i}^{*} \rangle))$.
\end{definicja}
\begin{twierdzenie}
    Każda czysta równowaga Nasha $s^{*} \in S$ jest mieszaną równowagą Nasha.
\end{twierdzenie}
\begin{definicja}
    \textbf{Nośnikiem} rozkładu łącznego strategii mieszanej $\pi$ nazywamy zbiór
    $S(\pi) = \{s \in S : \pi(s) > 0\}$.
    \textbf{Nośnikiem} rozkładu strategii mieszanej gracza $\pi_i$ nazywamy zbiór
    $S(\pi_i) = \{s_i \in S_i :\pi_i(s_i) > 0\}$.
\end{definicja}
\begin{twierdzenie}
    Niech $\pi^{*} \in \Delta$ będzie \textbf{MNE}.
    Wtedy $(\forall s_i \in S(\pi_{i}^{*}))(u_i(\langle s_i, \pi_{-i}^{*} \rangle) = u_i(\pi))$
\end{twierdzenie}