%%%% Uniwersalny Szablon Infy WPPT v4.2 (02.02.2018) %%%%

% uklad dokumentu
\documentclass[8pt]{extarticle}
\usepackage{xparse}
\usepackage[margin=0pt]{geometry}
\usepackage{enumerate} 
\frenchspacing
\linespread{0.5}
 \setlength{\parindent}{0pt}

% pakiety matematyczne
\usepackage{amssymb}   
\usepackage{amsthm}
\usepackage{amsmath}
\usepackage{amsfonts}
\usepackage{tikz}
\def\thm@space@setup{\thm@preskip=0pt
\thm@postskip=0pt}
% jezyk polski
\usepackage[polish]{babel}
\usepackage[utf8]{inputenc}
\usepackage{polski}
\usepackage[T1]{fontenc}

% hiperlacza
\usepackage{hyperref}
\hypersetup{
    colorlinks,
    citecolor=black,
    filecolor=black,
    linkcolor=black,
    urlcolor=black
}

\newtheoremstyle{mystyle}
  {1pt}%   Space above
  {1pt}% 
  {\slshape}%  Body font
  {}%          Indent amount (empty = no indent, \parindent = para indent)
  {\bfseries}% Thm head font
  {.}%         Punctuation after thm head
  {0.5em}%     Space after thm head: " " = normal interword space;
     %         \newline = linebreak
  {}%          Thm head spec (can be left empty, meaning `normal')

% wstawianie zdjec
\usepackage{graphicx} 

% deklaracja zadania
\theoremstyle{mystyle}
\newtheorem{zadanie}{Zadanie}[subsection]


% deklaracja metody
% \theoremstyle{remark}
\newtheorem*{metoda}{Metoda}
% \theoremstyle{plain}

% deklaracja rozwiazania
% \theoremstyle{remark}
\newtheorem*{rozwiazanie}{Rozwiązanie}
% \theoremstyle{plain}

\newtheorem{definicja}{Def}
\newtheorem{twierdzenie}{Tw}
\newtheorem{fakt}{F}

% zestaw - mpis
\newcommand{\EE}{\mathbb{E}}
\newcommand{\PP}{\mathbb{P}}
\newcommand{\Var}{\mathrm{Var}}
\newcommand{\DD}{\mathbb{D}}
\newcommand{\Cal}{\mathcal}
\newcommand{\bb}{\mathbb}
\newcommand{\red}[1]{\textcolor{red}{#1}}

% zestaw - programowanie
\usepackage{listings} 
\usepackage{minibox}
\usepackage{minted}
\usemintedstyle{borland}

% zestaw - akiso
%\usepackage{karnaugh-map} %%% nie ma w bazie overleaf
\usepackage{karnaughmap}
\usepackage{circuitikz}
\usepackage[inline]{enumitem}
\usepackage{tikz}
\usetikzlibrary{automata,positioning}

% zestaw - dyskretna
\newcommand{\stirf}[2]{\genfrac{[}{]}{0pt}{}{#1}{#2}}
\newcommand{\stirs}[2]{\genfrac{\{}{\}}{0pt}{}{#1}{#2}}

% zestaw - jftt
\newcommand{\oor}{\mathop{|}}
\DeclareMathOperator{\Lead}{Leading}
\DeclareMathOperator{\Trail}{Trailing}
\DeclareMathOperator{\fst}{First}
\DeclareMathOperator{\fol}{Follow}



\usepackage{multicol}
\usepackage{multirow}
\usepackage{float}
\setlength{\columnseprule}{1pt}
\def\columnseprulecolor{\color{black}}
\begin{document}

\begin{multicols*}{4}
    \begin{definicja}{Przestrzeń topologiczna}
        Para $(X, \Cal{O})$, gdzie $X$ jest zbiorem, a $\Cal{O} \subset \Cal{P}(X)$ (\textbf{topologia}):
        \begin{enumerate*}[label=\roman*]
            \item $\emptyset, X \in \Cal{O}$,
            \item $U, V \in \Cal{O} \Rightarrow U \cap V \in \Cal{O}$,
            \item $\Cal{T} \subset \Cal{O} \Rightarrow \bigcup\Cal{T} \in \Cal{O}$
        \end{enumerate*}
        Elementy $\Cal{O}$ nazywamy \textbf{zbiorami otwartymi}. Ich dopełnienia w $X$ nazywamy \textbf{zbiorami domkniętymi}.
    \end{definicja}

    \begin{definicja}
        W przestrzeni topologicznej $(X, \Cal{O})$ mówimy, że funkcja $f$ jest ciągła $\Leftrightarrow$ $\forall_{(\Cal{U} \in \Cal{O})}(f^{-1}[\Cal{U}] \in \Cal{O})$.
    \end{definicja}

    \begin{definicja}
        Ciąg ${(x_{n})}_{n}$ w przestrzeni 
        metrycznej 
        $(X, d)$ nazywamy \textbf{ciągiem Cauchy'ego}, 
        gdy:
        $((\forall \varepsilon > 0)(\exists N_0)(\forall n, m > N_0) d(x_n, x_m) < \varepsilon)$
        Mówimy, że prz.m. jest \textbf{zupełna}, jeśli każdy ciąg Cauchy'ego jest zbieżny.
    \end{definicja}

    \begin{twierdzenie}{Banach, o punkcie stałym, kontrakcji}
        Niech $(X, d)$ zupełna prz.m., $f: X \rightarrow X$ będzie \textbf{kontrakcją/odwzorowaniem zwężającym} ze stałą $\alpha < 1$.
        $(\forall x, y \in X)d(f(x), f(y)) < \alpha\cdot d(x, y)$. Wtedy $f$ ma dokładnije jeden punkt stały $x^*$, $f(x^*)=x^*$.
        Jeśli $x_0 \in X$ jest dowolnym elementem $X$, to ciąg $x_{n+1} = f(x_n)$, zbiega do $x^*$.
    \end{twierdzenie}

    \begin{definicja}
        Podzbiór przestrzeni topologicznej $(X, \Cal{D})$ nazywamy \textbf{zwartym}, jeśli z dowolnego pokrycia ${\{U_t\}}_{t\in T}$
        $X$ zbiorami otwartmi, można wybrać podpokrycie ${\{U_{t_i}\}}_{i = 1}^{n}$ skończone.
    \end{definicja}

    \begin{fakt}
        W przestrzeni $\bb{R}^k$ z naturalną topologią, zwartość podprzestrzeni jest równowazna z jej domkniętością i ograniczoniością.
    \end{fakt}

    \begin{fakt}
        W prz.m. zwartość zbioru $K$ jest równoznaczna z tym, że dla każdego ciągu w $K$, istnieje podciąg zbieżny do pewnego punktu $x \in K$.
    \end{fakt}

    \begin{fakt}
        W prz.m. każdy zbiór zwarty jest zupełny.
    \end{fakt}

    \begin{twierdzenie}
        Niech $K$, będzie zbiorem zwartym. Każda funkcja ciągła $f: K \rightarrow R$ osiąga swoje kresy (maksima i minima o ile istnieją).
    \end{twierdzenie}

    \begin{definicja}
        W przestrzeni liniowej $\bb{V}$ nad ciałem $\bb{R}$, niech $n \in \bb{N}, x_i \in \bb{V}$ i $\lambda_i \in K$, dla $i \in [n]$,
        będą takie, że $\sum\limits_{i=1}^{n}\lambda_i = 1$, gdzie $(\forall i \in [n])(\lambda_i \int [0, 1])$. Mówimy wtedy, że punkt postaci
        $\sum\limits_{i=0}^{n} \lambda_i x_i \in \bb{V}$ jest \textbf{kobinacją wypukłą} punktów ${(x_i)}^{n}_{i=1}$ z wagami ${(\lambda_i)}^{n}_{i=1}$.
    \end{definicja}

    \begin{definicja}
        Podzbiór prz. l. $X \subset \bb{V}$ jest \textbf{wypukły}, jeśli kombinacja wypukład dwóch doowlnych punktów z $X$ jest elementem $X$.
        W przeciwnym razie, jest \textbf{wklęsły}.
    \end{definicja}

    \begin{definicja}
        $X \subset \bb{V}$. $f: X \rightarrow \bb{R}$ jest \textbf{wypukła} gdy: 
        $(\forall x, y \in X)(\forall \lambda \in [0, 1])((\lambda x + (1 - \lambda)y \in X) \Rightarrow ((\lambda f(x) + (1 - \lambda)f(y)) \geq f(\lambda x + ( 1 - \lambda)y)))$
    \end{definicja}

    \begin{definicja}
        Zbiór wektorów ${(x_i)}^{k}_{i=0}$ w prz.l. $\bb{V}$ jest \textbf{afnicznie niezależny wzlędem} $x_0$ jeśli ${(x_i - x_0)}_{i=0}^{k}$ 
        jest liniowo neizależny od $\bb{V}$.
    \end{definicja}

    \begin{definicja}
        \textbf{Sympleksem k-wymiarowym} nazywamy najmniejszy możliwy podzbiór prz.l. $\bb{V}$ nad ciałem $\bb{R}$ taki,
        że zawiera wszystkie możliwe kombinacje wypukłe pewnych elem. ${(x_i)}_{i=0}^{k} \subset \bb{V}$,
         afnicznie niezależny wzgl. $x_0$.
        Elementy te nazywamy \textbf{wierzchołkami} sympleksu.
    \end{definicja}

    \begin{twierdzenie}
        Jeśli $S$ jest sympleksem, a $f: S \rightarrow S$ jest odwzorowaniem ciągłym, to $f$ ma punkt stały.
    \end{twierdzenie}

    \begin{definicja}
        W prz. m. $(X, d)$, \textbf{średnicą zbioru} $A \subset X$ nazywamy $diam(A) := \sup{\{d(x,y)\;:\; x, y \in A\}}$
    \end{definicja}

    \begin{definicja}
        Mówimy, że $f: X \rightarrow \bb{R}$ ma własność \textbf{wykresu domkniętego}, jeśli dla ciągu ${(x_i)_{i=1}^{\infty}}$, 
        zbieżnego do $x^*$, dowolny ciąg ${(y_i)_{i=1}^{\infty}}$, zbieżny do $y^*$, spełnienie warunku $(\forall i \in \bb{N})(y_i \in f(x_i))$,
        pociąga za sobą, że $y^* \in f(x^*)$.
    \end{definicja}

    \begin{twierdzenie}
        Niech $S$ będzie sympleksem $k$-wymiarowym, a $f: S \rightarrow \cal{P}(S)$ będzie miał własność wykresu domkniętego
        oraz spełnia warunek $(\forall s \in S)(f(s) \text{ jest zwarty i wypukły})$.
        Istnieje wtedy $x^* \in S$, że $x^* \in f(x^*)$.
    \end{twierdzenie}
\end{multicols*}

\end{document}